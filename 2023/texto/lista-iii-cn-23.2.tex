\documentclass[12pt,a4paper]{article}
\usepackage[utf8]{inputenc}
\usepackage[brazil]{babel}
\usepackage{graphicx}
\usepackage{amssymb, amsfonts, amsmath}
\usepackage{float}
\usepackage{enumerate}
\usepackage[top=2.5cm, bottom=2.5cm, left=1.25cm, right=1.25cm]{geometry}

\DeclareMathOperator{\sen}{sen}

\begin{document}
\pagestyle{empty}

\begin{center}
  \begin{tabular}{ccc}
    \begin{tabular}{c}
      \includegraphics[scale=0.25]{../../biblioteca/imagem/brasao-de-armas-brasil} \\
    \end{tabular} & 
    \begin{tabular}{c}
      Ministério da Educação \\
      Universidade Federal dos Vales do Jequitinhonha e Mucuri \\
      Faculdade de Ciências Sociais, Aplicadas e Exatas - FACSAE \\
      Departamento de Ciências Exatas - DCEX \\
      Disciplina: Cálculo Numérico \quad Semestre: 2023/2\\
      Prof. Dr. Luiz C. M. de Aquino\\
    \end{tabular} &
    \begin{tabular}{c}
      \includegraphics[scale=0.25]{../../biblioteca/imagem/logo-ufvjm} \\
    \end{tabular}
  \end{tabular}
\end{center}

\begin{center}
  \textbf{Lista III}
\end{center}

\begin{enumerate}
  
  \item Resolva o sistema abaixo utilizando o Método de Eliminação Gaussiana.

  $$%x = -1, y = 1, z = 2.
   \begin{cases}
    3x - 5y + z = -6 \\
    -x + y + 3z = 8 \\
    -7x + 3y - 6z = -2
   \end{cases}
  $$

 
  \item Resolva o sistema de equações abaixo de duas formas: pelo método de Gauss-Jacobi; pelo
  método de Gauss-Seidel. (Observação: considere uma tolerância de $10^{-5}$.)
  $$%x = 2/3, y = 7/6, z = 3/7, w = 1/9.
   \begin{cases}
    30x - 12y + 7z - 9w = 8 \\
    12x + 48y - 21z + 9w = 56 \\
    3x + 30y - 105z + 54w = -2 \\
    12x - 6y - 7z + 27w = 1
   \end{cases}
  $$

\end{enumerate}

\end{document}