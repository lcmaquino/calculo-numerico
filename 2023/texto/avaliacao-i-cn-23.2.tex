\documentclass[12pt,a4paper]{article}
\usepackage[utf8]{inputenc}
\usepackage[brazil]{babel}
\usepackage{graphicx}
\usepackage{amssymb, amsfonts, amsmath}
\usepackage{float}
\usepackage{enumerate}
\usepackage[top=2.5cm, bottom=2.5cm, left=1.25cm, right=1.25cm]{geometry}

\begin{document}
\pagestyle{empty}

\begin{center}
  \begin{tabular}{ccc}
    \begin{tabular}{c}
      \includegraphics[scale=0.25]{../../biblioteca/imagem/brasao-de-armas-brasil} \\
    \end{tabular} & 
    \begin{tabular}{c}
      Ministério da Educação \\
      Universidade Federal dos Vales do Jequitinhonha e Mucuri \\
      Faculdade de Ciências Sociais, Aplicadas e Exatas - FACSAE \\
      Departamento de Ciências Exatas - DCEX \\
      Disciplina: Cálculo Numérico \quad Semestre: 2023/2\\
      Prof. Dr. Luiz C. M. de Aquino\\
      Aluno(a):\rule{6cm}{0.1mm} \quad Data: \rule{0.5cm}{0.1mm}/\rule{0.5cm}{0.1mm}/\rule{1cm}{0.1mm}\\
    \end{tabular} &
    \begin{tabular}{c}
      \includegraphics[scale=0.25]{../../biblioteca/imagem/logo-ufvjm} \\
    \end{tabular}
  \end{tabular}
\end{center}

\begin{center}
 \textbf{Avaliação I}
\end{center}

\textbf{Instruções}
\begin{itemize}
 \item Todas as justificativas necessárias na solução de cada questão devem 
 estar presentes nesta avaliação;
 \item As respostas finais de cada questão devem estar escritas de caneta;
 \item Esta avaliação tem um total de 30,0 pontos.
\end{itemize}

\begin{enumerate}
  \item \textbf{[6,0 pontos]} Deduza a expressão para o termo $x_n$ no método de Newton:
  $$x_n = x_{n-1} - \frac{f(x_{n-1})}{f'(x_{n-1})}$$

  \item \textbf{[6,0 pontos]} Dê exemplo de uma função contínua que possua uma única raiz
  no intervalo $[2; 4]$, mas para a qual não é possível aplicar o Método de Newton para
  aproximar o valor dessa raiz. Justifique porque não é possível usar o método no seu exemplo.

  \item \textbf{[6,0 pontos]} Seja $a$ um número natural qualquer. Considere que $n$ seja um
  quadrado perfeito mais próximo de $a$. Utilize o método de Newton para provar que 
  $\sqrt{a}\approx \dfrac{a+n}{2\sqrt{n}}$. (Observação: dizemos que $n$ é um quadrado perfeito 
  se existe um natural $m$ tal que $n = m^2$.)
 
  \item \textbf{[6,0 pontos]} Use o método de Newton para calcular $x_1$ como aproximação 
  de $\sqrt[3]{3}$ considerando $x_0 = 1,5$.

  \item \textbf{[6,0 pontos]} Escreva a expressão para o termo $x_n$ do método de Newton
  para aproximar o ponto de interseção entre os gráficos de $f(x) = \dfrac{1}{x}$ e $g(x) = x^3 + 1$
  no primeiro quadrante.

\end{enumerate}

\end{document}
