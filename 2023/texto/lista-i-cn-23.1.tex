\documentclass[12pt,a4paper]{article}
\usepackage[utf8]{inputenc}
\usepackage[brazil]{babel}
\usepackage{graphicx}
\usepackage{amssymb, amsfonts, amsmath}
\usepackage{float}
\usepackage{enumerate}
\usepackage[top=2.5cm, bottom=2.5cm, left=1.25cm, right=1.25cm]{geometry}

\begin{document}
\pagestyle{empty}

\begin{center}
  \begin{tabular}{ccc}
    \begin{tabular}{c}
      \includegraphics[scale=0.25]{../../biblioteca/imagem/brasao-de-armas-brasil} \\
    \end{tabular} & 
    \begin{tabular}{c}
      Ministério da Educação \\
      Universidade Federal dos Vales do Jequitinhonha e Mucuri \\
      Faculdade de Ciências Sociais, Aplicadas e Exatas - FACSAE \\
      Departamento de Ciências Exatas - DCEX \\
      Disciplina: Cálculo Numérico \quad Semestre: 2022/2\\
      Prof. Dr. Luiz C. M. de Aquino\\
    \end{tabular} &
    \begin{tabular}{c}
      \includegraphics[scale=0.25]{../../biblioteca/imagem/logo-ufvjm} \\
    \end{tabular}
  \end{tabular}
\end{center}

\begin{center}
  \textbf{Lista I}
\end{center}

\begin{enumerate}
\item Use o Método da Bisseção para encontrar uma solução aproximada das seguintes equações (considere uma tolerância de $10^{-4}$):
 \begin{enumerate}
  \item $\cos\left(2^x\right) = \dfrac{1}{5}2^x$.
  \item $x^3 - \sqrt{2}x^2 + x - \sqrt{2} = 0$.
 \end{enumerate}

 \item Utilize os conhecimentos de Cálculo para provar que os gráficos das funções definidas por 
$f(x) = \cos\left(x^2\right)$ e $g(x) = x^3$ possuem um único ponto de interseção. Em seguida, 
de alguma maneria utilize o Método da Bisseção para determinar de modo aproximado esse ponto 
(considere uma tolerância de $10^{-4}$).

 \item Dado $a\in\mathbb{R}_+^*$ proponha uma maneira de usar o Método da Bisseção para calcular um valor aproximado de $\sqrt{a}$ com tolerância de $10^{-5}$. Em seguida, use a sua 
proposta para calcular o valor aproximado de $\sqrt{2}$.

 \item Invente uma equação que envolva termos exponenciais e trigonométricos e cuja solução seja $x = 4$. Em seguida, escolha um intervalo 
contendo $x = 4$ considerando que o Método da Bisseção será aplicado para encontrar uma solução 
aproximada da equação. O intervalo escolhido não deve ter o número $4$ no seu centro. 
Faça uma estimativa do número de passos do método que 
serão necessários para obter a precisão de $\varepsilon = 10^{-5}$. Execute essa quantidade de passos e compare a solução 
aproximada com a solução exata da equação.

 \item Seja a função definida por $f(t) =  -\dfrac{112}{9}t^3 + \dfrac{536}{9}t^2 -\dfrac{815}{9}t + \dfrac{400}{9}$. Verifique que $\bar{t} = \dfrac{5}{4}$ é solução de $f(t) = 0$. 
Em seguida, justifique porque não é possível utilizar o Método da Bisseção para determinar uma solução aproximada de $\bar{t}$.

\end{enumerate}

\begin{center}
\textbf{Gabarito}
\end{center}
\textbf{[1]} (a) $x\approx 0,3856201171875$. (b) $x\approx 1,41418457031250$. 
\textbf{[2]} Sugestão: considerando $h(x) = f(x) - g(x)$, analise o valor de 
$h(0)h(1)$ e de $h'$ em $[0;\,1]$. Ponto de interseção aproximado: 

$(0,889282226562501;\,0,703264730191224)$.
\textbf{[3]} Sugestão: note que $\sqrt{a}$ é a raiz de $x^2 - a = 0$ no intervalo $[0;\,a+1]$. Observação: este exercício admite outras respostas válidas. 
\textbf{[4]} Observação: este exercício admite várias respostas válidas.
\textbf{[5]} De fato, basta verificar que $f\left(\dfrac{5}{4}\right) = 0$. Não é possível, pois $f^\prime\left(\dfrac{5}{4}\right) = 0$ e 
$f^{\prime\prime}\left(\dfrac{5}{4}\right) > 0$.
\end{document}