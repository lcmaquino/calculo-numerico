\documentclass[12pt,a4paper]{article}
\usepackage[utf8]{inputenc}
\usepackage[brazil]{babel}
\usepackage{graphicx}
\usepackage{amssymb, amsfonts, amsmath}
\usepackage{float}
\usepackage{enumerate}
\usepackage[top=2.5cm, bottom=2.5cm, left=1.25cm, right=1.25cm]{geometry}

\DeclareMathOperator{\sen}{sen}

\begin{document}
\pagestyle{empty}

\begin{center}
  \begin{tabular}{ccc}
    \begin{tabular}{c}
      \includegraphics[scale=0.25]{../../biblioteca/imagem/brasao-de-armas-brasil} \\
    \end{tabular} & 
    \begin{tabular}{c}
      Ministério da Educação \\
      Universidade Federal dos Vales do Jequitinhonha e Mucuri \\
      Faculdade de Ciências Sociais, Aplicadas e Exatas - FACSAE \\
      Departamento de Ciências Exatas - DCEX \\
      Disciplina: Cálculo Numérico \quad Semestre: 2023/2\\
      Prof. Dr. Luiz C. M. de Aquino\\
    \end{tabular} &
    \begin{tabular}{c}
      \includegraphics[scale=0.25]{../../biblioteca/imagem/logo-ufvjm} \\
    \end{tabular}
  \end{tabular}
\end{center}

\begin{center}
  \textbf{Lista IV}
\end{center}

\begin{enumerate}
  
  \item Seja uma função $f$ da qual são conhecidos os valores descritos na
  tabela abaixo.
  %p(x) = 2*x^3 + x^2 - 5
  \begin{center}
    \begin{tabular}{c|c|c|c|c}
      $x_i$ & 1 & 1,5 & 2 & 2,5\\ \hline
      $f(x_i)$ & $-2$ & 4 & 15 & 32
    \end{tabular}
  \end{center}

  Determine o polinômio $p$ que interpola $f$ utilizando duas maneiras:
  \begin{enumerate}
   \item escrevendo $p$ na Forma de Lagrange;
   \item escrevendo $p$ na Forma de Newton.
  \end{enumerate}

  \item Seja uma função $f$ da qual são conhecidos os pontos $(x_0,\,f(x_0))$ e 
  $(x_1,\,f(x_1))$. Considere que $L(x)$ seja o polinômio na Forma de Lagrange que
  interpola $f$. Além disso, considere que $N(x)$ seja o polinômio na Forma de Newton que
  interpola $f$. Prove que $L(x)$ e $N(x)$ representam um mesmo polinômio.

  \item Seja $p$ o polinômio na Forma de Lagrange que interpola os pontos
  $(x_0,\,y_0)$, $(x_1,\,y_1)$, \ldots, $(x_n,\,y_n)$. Vamos definir o polinômio
  $$q(x) = \prod_{i=0}^{n} (x-x_i).$$ Prove que $p$ pode ser escrito no seguinte
  formato: 
  $$p(x) = \sum_{i=0}^n\frac{q(x)}{(x-x_i)q'(x_i)}y_i.$$

\end{enumerate}

\end{document}