\documentclass[12pt,a4paper]{article}
\usepackage[utf8]{inputenc}
\usepackage[brazil]{babel}
\usepackage{graphicx}
\usepackage{amssymb, amsfonts, amsmath}
\usepackage{float}
\usepackage{enumerate}
\usepackage[top=2.5cm, bottom=2.5cm, left=1.25cm, right=1.25cm]{geometry}

\begin{document}
\pagestyle{empty}

\begin{center}
  \begin{tabular}{ccc}
    \begin{tabular}{c}
      \includegraphics[scale=0.25]{../../biblioteca/imagem/brasao-de-armas-brasil} \\
    \end{tabular} & 
    \begin{tabular}{c}
      Ministério da Educação \\
      Universidade Federal dos Vales do Jequitinhonha e Mucuri \\
      Faculdade de Ciências Sociais, Aplicadas e Exatas - FACSAE \\
      Departamento de Ciências Exatas - DCEX \\
      Disciplina: Cálculo Numérico \quad Semestre: 2022/2\\
      Prof. Dr. Luiz C. M. de Aquino\\
      Aluno(a):\rule{6cm}{0.1mm} \quad Data: \rule{0.5cm}{0.1mm}/\rule{0.5cm}{0.1mm}/\rule{1cm}{0.1mm}\\
    \end{tabular} &
    \begin{tabular}{c}
      \includegraphics[scale=0.25]{../../biblioteca/imagem/logo-ufvjm} \\
    \end{tabular}
  \end{tabular}
\end{center}

\begin{center}
 \textbf{Avaliação II}
\end{center}

\textbf{Instruções}
\begin{itemize}
 \item Todas as justificativas necessárias na solução de cada questão devem 
 estar presentes nesta avaliação;
 \item As respostas finais de cada questão devem estar escritas de caneta;
 \item Esta avaliação tem um total de 35,0 pontos.
\end{itemize}

\begin{enumerate}
  \item \textbf{[9,0 pontos]} Suponha que a fatoração $LU$ de uma matriz $A$ seja conhecida.
  Explique como usar essa informação para resolver o sistema $Ax = b$. 

  \item \textbf{[9,0 pontos]} Arrume o sistema de equações abaixo de modo que ele seja
  resolvido: pelo método de Gauss-Jacobi; pelo método de Gauss-Seidel.

  $$
   \begin{cases}
    30x - 12y + 7z - 9w = 8 \\
    12x + 48y - 21z + 9w = 56 \\
    3x + 30y - 105z + 54w = -2 \\
    12x - 6y - 7z + 27w = 1
   \end{cases}
  $$

  \item \textbf{[8,0 pontos]} Seja uma função $f$ da qual são conhecidos os valores descritos na
  tabela abaixo.
  %p(x) = 2*x^3 + x^2 - 6
  \begin{center}
    \begin{tabular}{c|c|c|c|c}
      $x_i$ & 1 & 1,5 & 2 & 2,5\\ \hline
      $f(x_i)$ & $-3$ & 3 & 14 & 31
    \end{tabular}
  \end{center}

  Determine o polinômio $p$ que interpola $f$ utilizando duas maneiras:
  \begin{enumerate}
   \item escrevendo $p$ na Forma de Lagrange;
   \item escrevendo $p$ na Forma de Newton.
  \end{enumerate}

  \item \textbf{[9,0 pontos]} Seja $p$ o polinômio na Forma de Lagrange que interpola os pontos
  $(x_0,\,y_0)$, $(x_1,\,y_1)$, \ldots, $(x_n,\,y_n)$. Vamos definir o polinômio
  $$q(x) = \prod_{i=0}^{n} (x-x_i).$$ Prove que $p$ pode ser escrito no seguinte
  formato: 
  $$p(x) = \sum_{i=0}^n\frac{q(x)}{(x-x_i)q'(x_i)}y_i.$$
\end{enumerate}

\end{document}
