\documentclass[12pt,a4paper]{article}
\usepackage[utf8]{inputenc}
\usepackage[brazil]{babel}
\usepackage{graphicx}
\usepackage{amssymb, amsfonts, amsmath}
\usepackage{float}
\usepackage{enumerate}
\usepackage[top=2.5cm, bottom=2.5cm, left=1.25cm, right=1.25cm]{geometry}

\begin{document}
\pagestyle{empty}

\begin{center}
  \begin{tabular}{ccc}
    \begin{tabular}{c}
      \includegraphics[scale=0.25]{../../biblioteca/imagem/brasao-de-armas-brasil} \\
    \end{tabular} & 
    \begin{tabular}{c}
      Ministério da Educação \\
      Universidade Federal dos Vales do Jequitinhonha e Mucuri \\
      Faculdade de Ciências Sociais, Aplicadas e Exatas - FACSAE \\
      Departamento de Ciências Exatas - DCEX \\
      Disciplina: Cálculo Numérico \quad Semestre: 2022/2\\
      Prof. Dr. Luiz C. M. de Aquino\\
      Aluno(a):\rule{6cm}{0.1mm} \quad Data: \rule{0.5cm}{0.1mm}/\rule{0.5cm}{0.1mm}/\rule{1cm}{0.1mm}\\
    \end{tabular} &
    \begin{tabular}{c}
      \includegraphics[scale=0.25]{../../biblioteca/imagem/logo-ufvjm} \\
    \end{tabular}
  \end{tabular}
\end{center}

\begin{center}
 \textbf{Avaliação I}
\end{center}

\textbf{Instruções}
\begin{itemize}
 \item Todas as justificativas necessárias na solução de cada questão devem 
 estar presentes nesta avaliação;
 \item As respostas finais de cada questão devem estar escritas de caneta;
 \item Esta avaliação tem um total de 30,0 pontos.
\end{itemize}

\begin{enumerate}
  \item \textbf{[6,0 pontos]} Use o Método da Bisseção para encontrar uma solução aproximada 
  da seguinte equação (considere uma tolerância de $10^{-4}$):
    \begin{enumerate}
      \item $\cos\left(3^x\right) = \dfrac{1}{8}3^x$, no intervalo $[0,\, 1]$.
   \end{enumerate}
  
  \item \textbf{[6,0 pontos]} Utilize o Método de Newnton para determinar uma aproximação 
  para a raiz da função polinomial definida por $p(x) = 2x^4 -2x^3 -22x^2 - 10x + 8$ no 
  intervalo $[0;\,1]$ (considere uma tolerância de $10^{-5}$).
  
  \item \textbf{[6,0 pontos]} Dê exemplo de uma função contínua que possua uma única raiz no intervalo $[1; 3]$,
  mas para a qual não é possível aplicar o Método da Secante para aproximar essa raiz usando
  os chutes iniciais $x_0 = 1,2$ e $x_1 = 2,8$. Justifique porque não é possível usar
  o método no seu exemplo.

  \item \textbf{[6,0 pontos]} Seja $x$ um número natural qualquer. Considere que $n$ seja um
  quadrado perfeito mais próximo de $x$. Prove que $\sqrt{x}\approx \dfrac{x+n}{2\sqrt{n}}$.
  (Observação: dizemos que $n$ é um quadrado perfeito se existe um natural $m$ tal que $n = m^2$.)

  \item \textbf{[6,0 pontos]} Explique como obter a expressão para o termo $x_n$ da sequência
  definida pelo Método das Cordas para uma função $f$ contínua no intervalo $[a;\,b]$ e tal
  que $f(a)f(b)<0$.

\end{enumerate}

\end{document}
