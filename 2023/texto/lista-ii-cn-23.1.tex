\documentclass[12pt,a4paper]{article}
\usepackage[utf8]{inputenc}
\usepackage[brazil]{babel}
\usepackage{graphicx}
\usepackage{amssymb, amsfonts, amsmath}
\usepackage{float}
\usepackage{enumerate}
\usepackage[top=2.5cm, bottom=2.5cm, left=1.25cm, right=1.25cm]{geometry}

\DeclareMathOperator{\sen}{sen}

\begin{document}
\pagestyle{empty}

\begin{center}
  \begin{tabular}{ccc}
    \begin{tabular}{c}
      \includegraphics[scale=0.25]{../../biblioteca/imagem/brasao-de-armas-brasil} \\
    \end{tabular} & 
    \begin{tabular}{c}
      Ministério da Educação \\
      Universidade Federal dos Vales do Jequitinhonha e Mucuri \\
      Faculdade de Ciências Sociais, Aplicadas e Exatas - FACSAE \\
      Departamento de Ciências Exatas - DCEX \\
      Disciplina: Cálculo Numérico \quad Semestre: 2022/2\\
      Prof. Dr. Luiz C. M. de Aquino\\
    \end{tabular} &
    \begin{tabular}{c}
      \includegraphics[scale=0.25]{../../biblioteca/imagem/logo-ufvjm} \\
    \end{tabular}
  \end{tabular}
\end{center}

\begin{center}
  \textbf{Lista II}
\end{center}

\begin{enumerate}
  \item Considere a função definida por $f(x) = \sen x - \dfrac{1}{5}$. Aplique o
  Método da Bisseção para determinar uma aproximação da raiz de $f$ no intervalo $[0;\,1]$,
  com tolerância de $10^{-4}$. Em seguida, aplique o Método das Cordas para também
  encontrar uma aproximação dessa raiz, considerando o mesmo intervalo e tolerância e
  usando $x_0=0,5$ como chute inicial. Comparando os dois métodos, houve alguma vantagem 
  em usar o Método das Cordas?

  \item Explique como obter a expressão para o termo $x_n$ da sequência definida
  pelo Método das Cordas para uma função $f$ contínua no intervalo $[a;\,b]$ e tal
  que $f(a)f(b)<0$.

  \item Dê exemplo de uma função contínua que possua uma única raiz no intervalo $[1; 3]$,
  mas para a qual não é possível aplicar o Método da Secante para aproximar essa raiz usando
  os chutes iniciais $x_0 = 1,4$ e $x_1 = 2,6$. Justifique porque não é possível usar
  o método no seu exemplo.
  
  \item Utilize o Método de Newnton para determinar uma aproximação para a raiz da
  função polinomial definida por $p(x) = 2x^4 -2x^3 -22x^2 - 10x + 8$ no intervalo
  $[0;\,1]$ (considere uma tolerância de $10^{-5}$).
  
  \item Seja $x$ um número natural qualquer. Considere que $n$ seja um quadrado
  perfeito mais próximo de $x$. Prove que $\sqrt{x}\approx \dfrac{x+n}{2\sqrt{n}}$. 
 (Observação: dizemos que $n$ é um quadrado perfeito se existe um natural $m$ tal
 que $n = m^2$.) 
\end{enumerate}

\begin{center}
  \textbf{Gabarito}
\end{center}

\textbf{[1]} Método da bisseção: $x\approx 0,201416015625$. Método das Cordas: 
$x\approx 0,201333044061041$. Comparando os métodos, a vantagem de usar o 
Método das Cordas foi executar menos passos para obter a aproximação desejada.
\textbf{[2]} Sugestão: Primeiro, determine a equação da reta que passa por $(x_n;\,f(x_n))$ e
é paralela a reta passando por $(a;\,f(a))$ e $(b;\,f(b))$. Em seguida, defina $x_{n+1}$ como
sendo a abscissa do ponto de interseção entre esta reta e o eixo $x$. 
\textbf{[3]} Sugestão: tente montar uma função de tal modo que $f(1,4) = f(2,6)$. 
Observação: esse exercício possui várias soluções. 
\textbf{[4]} $x\approx 0.41421$. 
\textbf{[5]} Sugestão: Aplique o Método de Newton na resolução aproximada (em $u$) da equação $u^2 - x = 0$. Use como valor inicial $u_0 = \sqrt{n}$.
\end{document}