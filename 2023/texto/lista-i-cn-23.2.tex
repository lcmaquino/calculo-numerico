\documentclass[12pt,a4paper]{article}
\usepackage[utf8]{inputenc}
\usepackage[brazil]{babel}
\usepackage{graphicx}
\usepackage{amssymb, amsfonts, amsmath}
\usepackage{float}
\usepackage{enumerate}
\usepackage[top=2.5cm, bottom=2.5cm, left=1.25cm, right=1.25cm]{geometry}

\begin{document}
\pagestyle{empty}

\begin{center}
  \begin{tabular}{ccc}
    \begin{tabular}{c}
      \includegraphics[scale=0.25]{../../biblioteca/imagem/brasao-de-armas-brasil} \\
    \end{tabular} & 
    \begin{tabular}{c}
      Ministério da Educação \\
      Universidade Federal dos Vales do Jequitinhonha e Mucuri \\
      Faculdade de Ciências Sociais, Aplicadas e Exatas - FACSAE \\
      Departamento de Ciências Exatas - DCEX \\
      Disciplina: Cálculo Numérico \quad Semestre: 2023/2\\
      Prof. Dr. Luiz C. M. de Aquino\\
    \end{tabular} &
    \begin{tabular}{c}
      \includegraphics[scale=0.25]{../../biblioteca/imagem/logo-ufvjm} \\
    \end{tabular}
  \end{tabular}
\end{center}

\begin{center}
  \textbf{Lista I}
\end{center}

\begin{enumerate}
\item Use o Método da Bisseção para encontrar uma solução aproximada das seguintes
equações (considere uma tolerância de $10^{-4}$):
 \begin{enumerate}
  \item $\cos\left(2^x\right) = \dfrac{1}{5}2^x$.
  \item $x^3 - \sqrt{2}x^2 + x - \sqrt{2} = 0$.
 \end{enumerate}

 \item Utilize os conhecimentos de Cálculo para provar que os gráficos das funções definidas por 
$f(x) = \cos\left(x^2\right)$ e $g(x) = x^3$ possuem um único ponto de interseção. Em seguida, utilize 
o Método da Bisseção para determinar de modo aproximado esse ponto (considere uma tolerância de $10^{-4}$).

  \item Dê exemplo de uma função contínua que possua uma única raiz no intervalo $[1; 3]$,
  mas para a qual não é possível aplicar o Método de Newton para aproximar o valor dessa
  raiz. Justifique porque não é possível usar o método no seu exemplo.
  
  \item Utilize o Método de Newnton para determinar uma aproximação para a raiz da
  função polinomial definida por $p(x) = 2x^4 -2x^3 -22x^2 - 10x + 8$ no intervalo
  $[0;\,1]$ (considere uma tolerância de $10^{-5}$).
  
  \item Seja $x$ um número natural qualquer. Considere que $n$ seja um quadrado
  perfeito mais próximo de $x$. Prove que $\sqrt{x}\approx \dfrac{x+n}{2\sqrt{n}}$. 
 (Observação: dizemos que $n$ é um quadrado perfeito se existe um natural $m$ tal
 que $n = m^2$.) 

\end{enumerate}

\begin{center}
\textbf{Gabarito}
\end{center}
\textbf{[1]} (a) $x\approx 0,3856201171875$. (b) $x\approx 1,41418457031250$. 
\textbf{[2]} Sugestão: considerando $h(x) = f(x) - g(x)$, analise o valor de 
$h(0)h(1)$ e de $h'$ em $[0;\,1]$. Ponto de interseção aproximado: 

$(0,889282226562501;\,0,703264730191224)$.
\textbf{[3]} Observação: esse exercício admite várias respostas.
\textbf{[4]} $x\approx 0,41421$. 
\textbf{[5]} Sugestão: aplique o Método de Newton na resolução aproximada (em $u$) da 
equação $u^2 - x = 0$. Use como valor inicial $u_0 = \sqrt{n}$.
\end{document}