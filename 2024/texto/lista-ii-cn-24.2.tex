\documentclass[12pt,a4paper]{article}
\usepackage[utf8]{inputenc}
\usepackage[brazil]{babel}
\usepackage{graphicx}
\usepackage{amssymb, amsfonts, amsmath}
\usepackage{float}
\usepackage{enumerate}
\usepackage[top=2.5cm, bottom=2.5cm, left=1.25cm, right=1.25cm]{geometry}

\begin{document}
\pagestyle{empty}

\begin{center}
  \begin{tabular}{ccc}
    \begin{tabular}{c}
      \includegraphics[scale=0.25]{../../biblioteca/imagem/brasao-de-armas-brasil} \\
    \end{tabular} & 
    \begin{tabular}{c}
      Ministério da Educação \\
      Universidade Federal dos Vales do Jequitinhonha e Mucuri \\
      Faculdade de Ciências Sociais, Aplicadas e Exatas - FACSAE \\
      Departamento de Ciências Exatas - DCEX \\
      Disciplina: Cálculo Numérico \quad Semestre: 2024/2\\
      Prof. Dr. Luiz C. M. de Aquino\\
    \end{tabular} &
    \begin{tabular}{c}
      \includegraphics[scale=0.25]{../../biblioteca/imagem/logo-ufvjm} \\
    \end{tabular}
  \end{tabular}
\end{center}

\begin{center}
  \textbf{Lista II}
\end{center}

\begin{enumerate}
 \item Use o Método da Falsa Posição para encontrar a raiz aproximada da equação
 $e^{x} - e^{-x} = 2\cos x$ no intervalo $[0;\,1]$
 (considere uma tolerância de $10^{-4}$).

  \item Dê exemplo de uma função contínua que possua uma única raiz no intervalo $[1; 3]$,
  mas para a qual não é possível aplicar o Método de Newton para aproximar essa raiz.
  Justifique porque não é possível usar o método no seu exemplo.
  
  \item Utilize o Método de Newnton para determinar uma aproximação para a raiz da
  função polinomial definida por $p(x) = 2x^4 -2x^3 -22x^2 - 10x + 8$ no intervalo
  $[0;\,1]$ (considere uma tolerância de $10^{-5}$).

  \item Utilize um método numérico para determinar aproximadamente qual é o ponto da
  circunferência $x^2 + y^2 = 1$ que está mais próximo da reta $x + 2y - 4 = 0$.

  \item Seja $x$ um número natural qualquer. Considere que $n$ seja um quadrado
  perfeito mais próximo de $x$. Prove que $\sqrt{x}\approx \dfrac{x+n}{2\sqrt{n}}$. 
 (Observação: dizemos que $n$ é um quadrado perfeito se existe um natural $m$ tal
 que $n = m^2$.) 

\end{enumerate}
  
\end{document}