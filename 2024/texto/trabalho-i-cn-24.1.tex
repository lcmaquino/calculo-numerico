\documentclass[12pt,a4paper]{article}
\usepackage[utf8]{inputenc}
\usepackage[brazil]{babel}
\usepackage{graphicx}
\usepackage{amssymb, amsfonts, amsmath}
\usepackage{float}
\usepackage{enumerate}
\usepackage[top=2.5cm, bottom=2.5cm, left=1.25cm, right=1.25cm]{geometry}

\DeclareMathOperator{\sen}{sen}

\begin{document}
\pagestyle{empty}

\begin{center}
  \begin{tabular}{ccc}
    \begin{tabular}{c}
      \includegraphics[scale=0.25]{../../biblioteca/imagem/brasao-de-armas-brasil} \\
    \end{tabular} & 
    \begin{tabular}{c}
      Ministério da Educação \\
      Universidade Federal dos Vales do Jequitinhonha e Mucuri \\
      Faculdade de Ciências Sociais, Aplicadas e Exatas - FACSAE \\
      Departamento de Ciências Exatas - DCEX \\
      Disciplina: Cálculo Numérico \quad Semestre: 2024/1\\
      Prof. Dr. Luiz C. M. de Aquino\\
    \end{tabular} &
    \begin{tabular}{c}
      \includegraphics[scale=0.25]{../../biblioteca/imagem/logo-ufvjm} \\
    \end{tabular}
  \end{tabular}
\end{center}

\begin{center}
  \textbf{Trabalho I}
\end{center}

\begin{enumerate}
  
  \item Considere a matriz $M$ abaixo:
  $$
    M=\begin{bmatrix}
        -2 & 4 & -3 & 4\\
        -1 & 1 & 2 & 4\\
        -1 & 2 & 2 & -1\\
        3 & 6 & -3 & -1
    \end{bmatrix}
  $$

  \begin{enumerate}
    \item Determine a fatoração LU da matriz $M$.
    \item Use a fatoração LU de $M$ para resolver os sistemas $Mu = b$ e $Mv = c$, sendo que
  $$
    b=\begin{bmatrix}
        1\\
        -2\\
        4\\
        1
    \end{bmatrix},\quad
    c=\begin{bmatrix}
        3\\
        2\\
        -2\\
        5
    \end{bmatrix}
  $$
  \end{enumerate}

\end{enumerate}

\end{document}