\documentclass[12pt,a4paper]{article}
\usepackage[utf8]{inputenc}
\usepackage[brazil]{babel}
\usepackage{graphicx}
\usepackage{amssymb, amsfonts, amsmath}
\usepackage{float}
\usepackage{enumerate}
\usepackage[top=2.5cm, bottom=2.5cm, left=1.25cm, right=1.25cm]{geometry}

\begin{document}
\pagestyle{empty}

\begin{center}
  \begin{tabular}{ccc}
    \begin{tabular}{c}
      \includegraphics[scale=0.25]{../../biblioteca/imagem/brasao-de-armas-brasil} \\
    \end{tabular} & 
    \begin{tabular}{c}
      Ministério da Educação \\
      Universidade Federal dos Vales do Jequitinhonha e Mucuri \\
      Faculdade de Ciências Sociais, Aplicadas e Exatas - FACSAE \\
      Departamento de Ciências Exatas - DCEX \\
      Disciplina: Cálculo Numérico \quad Semestre: 2024/2\\
      Prof. Dr. Luiz C. M. de Aquino\\
      Aluno(a):\rule{6cm}{0.1mm} \quad Data: \rule{0.5cm}{0.1mm}/\rule{0.5cm}{0.1mm}/\rule{1cm}{0.1mm}\\
    \end{tabular} &
    \begin{tabular}{c}
      \includegraphics[scale=0.25]{../../biblioteca/imagem/logo-ufvjm} \\
    \end{tabular}
  \end{tabular}
\end{center}

\begin{center}
 \textbf{Avaliação I - 2ª Chamada}
\end{center}

\textbf{Instruções}
\begin{itemize}
 \item Todas as justificativas necessárias na solução de cada questão devem 
 estar presentes nesta avaliação;
 \item As respostas finais de cada questão devem estar escritas de caneta;
 \item Esta avaliação tem um total de 30,0 pontos.
\end{itemize}

\begin{enumerate}
  \item \textbf{[6,0 pontos]} Utilize o Método da Bisseção para determinar de modo
  aproximado o ponto de interseção entre os gráficos das funções definidas por
  $f(x) = \cos\left(x^3\right)$ e $g(x) = 3^x - 3$. (Observação: considere uma
  tolerância de $10^{-4}$).

  \item \textbf{[8,0 pontos]} Suponha que o custo para produzir $x$ unidades de certo produto
  seja aproximadamente dado por $C(x) = \dfrac{1}{4}x^{\frac{2}{3}} + 20x + 900$. Se esse
  produto for vendido por R\$ 30,00 a unidade, então a partir de qual quantidade não
  haverá prejuízo? Observação: use o método de Newton na solução.

  \item \textbf{[8,0 pontos]} Use o método de Newton para calcular uma aproximação 
  de $\sqrt[3]{3}$. Considere uma tolerância de $10^{-4}$.

  \item \textbf{[8,0 pontos]} Utilize um método numérico para determinar aproximadamente qual é o ponto da
  circunferência $x^2 + y^2 = 1$ que está mais próximo da reta $2x + y - 4 = 0$.

\end{enumerate}

\end{document}
