\documentclass[12pt,a4paper]{article}
\usepackage[utf8]{inputenc}
\usepackage[brazil]{babel}
\usepackage{graphicx}
\usepackage{amssymb, amsfonts, amsmath}
\usepackage{float}
\usepackage{enumerate}
\usepackage[top=2.5cm, bottom=2.5cm, left=1.25cm, right=1.25cm]{geometry}

\DeclareMathOperator{\sen}{sen}

\begin{document}
\pagestyle{empty}

\begin{center}
  \begin{tabular}{ccc}
    \begin{tabular}{c}
      \includegraphics[scale=0.25]{../../biblioteca/imagem/brasao-de-armas-brasil} \\
    \end{tabular} & 
    \begin{tabular}{c}
      Ministério da Educação \\
      Universidade Federal dos Vales do Jequitinhonha e Mucuri \\
      Faculdade de Ciências Sociais, Aplicadas e Exatas - FACSAE \\
      Departamento de Ciências Exatas - DCEX \\
      Disciplina: Cálculo Numérico \quad Semestre: 2024/1\\
      Prof. Dr. Luiz C. M. de Aquino\\
    \end{tabular} &
    \begin{tabular}{c}
      \includegraphics[scale=0.25]{../../biblioteca/imagem/logo-ufvjm} \\
    \end{tabular}
  \end{tabular}
\end{center}

\begin{center}
  \textbf{Lista II}
\end{center}

\begin{enumerate}
  \item Use o método de Newton para calcular uma solução aproximada das equações abaixo.
  (Observação: considere uma precisão de $10^{-4}$.)
  \begin{enumerate}
    \item $x - \dfrac{1}{4}\sen x = \dfrac{\pi}{3}$
    \item $3^{-x} - x^2 = x$
  \end{enumerate}
  
  \item Suponha que a temperatura de um certo objeto após $t$ minutos deixado em um certo ambiente
  seja dada por:
  $$T(t) = 62e^{\alpha t} + 38$$
  
  Use o método de Newton para determinar o valor aproximado de $\alpha$ considerando que $T(10) = 60$.
  
\end{enumerate}

\begin{center}
  \textbf{Gabarito}
\end{center}
  \textbf{[1]} 
  (a) $x\approx 1,28721252248392$ 
  (b) $x\approx 0,433377193246367$ 
  \textbf{[2]} $\alpha \approx -0,103609193162839$

\end{document}