\documentclass[12pt,a4paper]{article}
\usepackage[utf8]{inputenc}
\usepackage[brazil]{babel}
\usepackage{graphicx}
\usepackage{amssymb, amsfonts, amsmath}
\usepackage{float}
\usepackage{enumerate}
\usepackage[top=2.5cm, bottom=2.5cm, left=1.25cm, right=1.25cm]{geometry}

\begin{document}
\pagestyle{empty}

\begin{center}
  \begin{tabular}{ccc}
    \begin{tabular}{c}
      \includegraphics[scale=0.25]{../../biblioteca/imagem/brasao-de-armas-brasil} \\
    \end{tabular} & 
    \begin{tabular}{c}
      Ministério da Educação \\
      Universidade Federal dos Vales do Jequitinhonha e Mucuri \\
      Faculdade de Ciências Sociais, Aplicadas e Exatas - FACSAE \\
      Departamento de Ciências Exatas - DCEX \\
      Disciplina: Cálculo Numérico \quad Semestre: 2024/2\\
      Prof. Dr. Luiz C. M. de Aquino\\
    \end{tabular} &
    \begin{tabular}{c}
      \includegraphics[scale=0.25]{../../biblioteca/imagem/logo-ufvjm} \\
    \end{tabular}
  \end{tabular}
\end{center}

\begin{center}
  \textbf{Lista I}
\end{center}

\begin{enumerate}
  \item Utilize os conhecimentos de Cálculo para provar que os gráficos das
  funções definidas por $f(x) = \cos\left(x^2\right)$ e $g(x) = x^3$ possuem
  um único ponto de interseção. Em seguida, de alguma maneria utilize o Método
  da Bisseção para determinar de modo aproximado esse ponto (considere uma
  tolerância de $10^{-4}$).

  \item Proponha uma maneira de calcular de modo aproximado o valor de $\sqrt[10]{10}$
  usando o Método da Bisseção. (Obs.: considere uma tolerância de $10^{-4}$)

\end{enumerate}
  
\end{document}