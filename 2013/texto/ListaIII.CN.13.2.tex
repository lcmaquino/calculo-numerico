\documentclass[12pt,a4paper]{article}
\usepackage[utf8]{inputenc}
\usepackage[brazil]{babel}
\usepackage{graphicx}
\usepackage{amssymb, amsfonts, amsmath}
\usepackage{color}
\usepackage{float}
\usepackage{enumerate}
%\usepackage{subfigure}
\usepackage[top=2.5cm, bottom=2.5cm, left=1.25cm, right=1.25cm]{geometry}

\newcommand{\sen}{\,\textrm{sen}\,}
\newcommand{\tg}{\,\textrm{tg}\,}

\begin{document}
\pagestyle{empty}

\begin{center}
\begin{tabular}{ccc}
\begin{tabular}{c}
\includegraphics[scale=0.25]{../../biblioteca/imagem/brasao-de-armas-brasil} \\
\end{tabular} & 
\begin{tabular}{c}
Ministério da Educação \\
Universidade Federal dos Vales do Jequitinhonha e Mucuri \\
Faculdade de Ciências Sociais, Aplicadas e Exatas - FACSAE \\
Departamento de Ciências Exatas - DCEX \\
Disciplina: Cálculo Numérico\\
Prof.: Luiz C. M. de Aquino\\
\end{tabular} &
\begin{tabular}{c}
\includegraphics[scale=0.25]{../../biblioteca/imagem/logo-ufvjm} \\
\end{tabular}
\end{tabular}
\end{center}

\begin{center}
 \textbf{Lista de Exercícios III}
\end{center}

\begin{enumerate}

 \item Utilize o Método de Newton para determinar uma aproximação para a raiz da função polinomial definida por $p(x) = 2x^4 -2x^3 -22x^2 - 10x + 8$ no intervalo $[0;\,1]$ (considere uma tolerância 
 de $10^{-5}$).
 \item Use o Método da Falsa Posição para encontrar a raiz aproximada da equação $e^{x} - e^{-x} = 2\cos x$ no intervalo $[0;\,1]$ (considere uma tolerância de $10^{-5}$).
 \item Considere o problema de encontrar uma raiz aproximada da equação $e^{-x^2} = \dfrac{1}{2}$ no intervalo $\left[\dfrac{1}{2};\,1\right]$. Determine uma função de iteração e resolva este problema pelo Método do Ponto Fixo.
 \item Seja $\phi$ uma função de iteração da equação $f(x) = 0$ no intervalo $[a;\, b]$. Prove que se $\phi$ é contínua em $[a;\,b]$ e $\phi(x)\in[a;\,b]$ para 
 todo $x\in[a;\, b]$, então $\phi$ possui algum ponto fixo em $[a;\,b]$.
 \item Seja $x$ um número natural não nulo qualquer. Considere que $n$ seja um quadrado perfeito mais próximo de $x$. Prove que $\sqrt{x}\approx \dfrac{x+n}{2\sqrt{n}}$. 
 (Observação: dizemos que $n$ é um quadrado perfeito se existe um natural $m$ tal que $n = m^2$.)

\end{enumerate}

\begin{center}
\textbf{Gabarito}
\end{center}
\textbf{[1]} $x\approx 0.41421$.
\textbf{[2]} $x\approx 0,70329$. 
\textbf{[3]} Considere $\phi(x) = x + e^{-x^2} - \dfrac{1}{2}$. Note que $|\phi'(x)| < 1$ para $x\in\left[\dfrac{1}{2};\,1\right]$. Deste modo, a sequência 
$x_{n+1}=\phi(x_n)$ é convergente e teremos $x\approx 0,832438$ como solução aproximada da equação dada.
\textbf{[4]} Sugestão: Usando que $\phi$ é contínua em $[a;\,b]$, justifique que a função definida por $g(x) = \phi(x) - x$ é contínua em $[a;\,b]$. Já usando que $\phi(x)\in[a;\,b]$ para todo $x\in[a;\,b]$, justifique que $g(a)g(b) < 0$. 
Aplicando então o Teorema do Valor Intermediário, conclua que existe $\alpha\in[a;\,b]$ tal que $g(\alpha)=0$. A partir disso, conclua que $\phi$ possui algum ponto fixo em $[a;\,b]$. 
\textbf{[5]} Sugestão: Aplique o Método de Newton na resolução aproximada (em $u$) da equação $u^2 - x = 0$. Use como valor inicial $u_0 = \sqrt{n}$.
\end{document}
