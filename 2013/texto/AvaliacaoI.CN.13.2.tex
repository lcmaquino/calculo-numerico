\documentclass[12pt,a4paper]{article}
\usepackage[utf8]{inputenc}
\usepackage[brazil]{babel}
\usepackage{graphicx}
\usepackage{amssymb, amsfonts, amsmath}
\usepackage{color}
\usepackage{float}
\usepackage{enumerate}
%\usepackage{subfigure}
\usepackage[top=2.5cm, bottom=2.5cm, left=1.25cm, right=1.25cm]{geometry}

\newcommand{\sen}{\,\textrm{sen}\,}
\newcommand{\tg}{\,\textrm{tg}\,}

\begin{document}
\pagestyle{empty}

\begin{center}
\begin{tabular}{ccc}
\begin{tabular}{c}
\includegraphics[scale=0.25]{../../biblioteca/imagem/brasao-de-armas-brasil} \\
\end{tabular} & 
\begin{tabular}{c}
Ministério da Educação \\
Universidade Federal dos Vales do Jequitinhonha e Mucuri \\
Faculdade de Ciências Sociais, Aplicadas e Exatas - FACSAE \\
Departamento de Ciências Exatas - DCEX \\
Disciplina: Cálculo Numérico\\
Prof.: Luiz C. M. de Aquino\\
\end{tabular} &
\begin{tabular}{c}
\includegraphics[scale=0.25]{../../biblioteca/imagem/logo-ufvjm} \\
\end{tabular}
\end{tabular}

Aluno(a): \rule{0.5\textwidth}{0.01cm} \quad Data: \rule{0.6cm}{0.01cm} / \rule{0.6cm}{0.01cm} / \rule{1.25cm}{0.01cm}
\end{center}

\begin{center}
 \textbf{Avaliação I}
\end{center}

\textbf{Instruções}
\begin{itemize}
 \item Todas as justificativas necessárias na solução de cada questão devem estar presentes nesta avaliação;
 \item As respostas finais de cada questão devem estar escritas de caneta;
 \item Esta avaliação tem um total de 25,0 pontos.
\end{itemize}

\begin{enumerate}
 \item \textbf{[6,0 pontos]} Dê exemplo de uma equação que envolva termos do tipo $2^u$ e $\sen u$ e cuja solução seja $x = 4$. Em seguida, determine um intervalo 
contendo $x = 4$ e considere que o Método da Bisseção será aplicado nesse intervalo. Faça uma estimativa do número de passos do método que 
serão necessários para obter a precisão de $\varepsilon = 10^{-2}$. Execute essa quantidade de passos e compare a solução aproximada com a solução 
exata da equação.
 \item \textbf{[4,5 pontos]} Seja $x$ um número natural não nulo qualquer. Considere que $n$ seja um quadrado perfeito mais próximo de $x$. Prove que $\sqrt{x}\approx \dfrac{x+n}{2\sqrt{n}}$. 
 (Observação: dizemos que $n$ é um quadrado perfeito se existe um natural $m$ tal que $n = m^2$.)
 
 \item \textbf{[4,5 pontos]} Considere o problema de encontrar uma raiz aproximada da equação $e^{-x^2} = \dfrac{1}{2}$ no intervalo $\left[\dfrac{1}{2};\,1\right]$. Determine uma função de iteração e resolva este problema pelo Método do Ponto Fixo. 
 (Considere uma tolerância de $\varepsilon = 10^{-2}$.)

 \item \textbf{[5,0 pontos]} Utilize o Método de Newton para determinar aproximadamente o ponto de mínimo da função definida por $f(x) = x^2 + \sen x$ no intervalo $[-1; 1]$. (Considere uma tolerância de $\varepsilon = 10^{-2}$.)

 \item \textbf{[5,0 pontos]} Seja $f:[0;\, 1]\to [0;\,1]$ uma função contínua em todo o seu domínio. Prove que para todo $n\in\mathbb{N}^*$, existe $\bar{x}\in[0;\,1]$ que é 
 solução da equação $f(x) = x^n$.

\end{enumerate}
\end{document}
