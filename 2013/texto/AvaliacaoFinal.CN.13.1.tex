\documentclass[11pt,a4paper]{article}
\usepackage[utf8]{inputenc}
\usepackage[brazil]{babel}
\usepackage{graphicx}
\usepackage{amssymb, amsfonts, amsmath}
\usepackage{color}
\usepackage{float}
\usepackage{enumerate}
%\usepackage{subfigure}
\usepackage[top=2.5cm, bottom=2.5cm, left=1.25cm, right=1.25cm]{geometry}

\newcommand{\sen}{\,\textrm{sen}\,}
\newcommand{\tg}{\,\textrm{tg}\,}

\begin{document}
\pagestyle{empty}

\begin{center}
\begin{tabular}{ccc}
\begin{tabular}{c}
\includegraphics[scale=0.25]{../../biblioteca/imagem/brasao-de-armas-brasil} \\
\end{tabular} & 
\begin{tabular}{c}
Ministério da Educação \\
Universidade Federal dos Vales do Jequitinhonha e Mucuri - UFVJM\\
Faculdade de Ciências Sociais, Aplicadas e Exatas - FACSAE \\
Departamento de Ciências Exatas - DCEX \\
Disciplina: Cálculo Numérico\\
Prof.: Luiz C. M. de Aquino\\
\end{tabular} &
\begin{tabular}{c}
\includegraphics[scale=0.25]{../../biblioteca/imagem/logo-ufvjm} \\
\end{tabular}
\end{tabular}

Aluno(a): \rule{0.5\textwidth}{0.01cm} \quad Data: \rule{0.6cm}{0.01cm} / \rule{0.6cm}{0.01cm} / \rule{1.25cm}{0.01cm}
\end{center}

\begin{center}
 \textbf{Avaliação Final}
\end{center}

\textbf{Instruções}
\begin{itemize}
 \item Todas as justificativas necessárias na solução de cada questão devem estar presentes nesta avaliação;
 \item As respostas finais de cada questão devem estar escritas de caneta;
 \item Cada questão vale 20 pontos, totalizando-se assim 100 pontos.
\end{itemize}

\begin{enumerate}
 
 \item Dado $a\in\mathbb{R}_+^*$ proponha uma maneira de usar o Método da Bisseção para calcular um valor aproximado de $\sqrt{a}$ com tolerância de $10^{-5}$.
 \item Seja $x$ um número natural qualquer. Considere que $n$ seja um quadrado perfeito mais próximo de $x$. Prove que $\sqrt{x}\approx \dfrac{x+n}{2\sqrt{n}}$. 
 (Observação: dizemos que $n$ é um quadrado perfeito se existe um natural $m$ tal que $n = m^2$.)
 \item A cada passo no Método da Falsa Posição, escolhemos $x_k = \dfrac{a_k|f(b_k)| + b_k|f(a_k)|}{|f(a_k)|+|f(b_k)|}$, sendo que no intervalo $[a_k;\,b_k]$ temos $f(a_k)f(b_k)<0$. Prove que 
 esta escolha de $x_k$ coincide com a abscissa do ponto de interseção entre o eixo $x$ e a reta passando por $(a_k,\,f(a_k))$ e $(b_k,\,f(b_k))$.
  
 \item Sobre certa função $f$ são conhecidos os pontos $(x_k,\,f(x_k))$, com $k=0$, $1$, $2$, \ldots, $n$. Suponha que seja 
aplicado o Método dos Mínimos Quadrados para determinar a função $\phi(x) = ag_1(x) + bg_2(x)$ que melhor se ajusta a $f$. Deduza que os coeficientes $a$ e 
$b$ são a solução do sistema de equações:
$$\begin{cases}
   c_{11}a + c_{12}b = d_1 \\
   c_{21}a + c_{22}b = d_2
  \end{cases},
$$
onde $\displaystyle c_{ij} = \sum_{k=0}^n g_i(x_k)g_j(x_k)$ e $\displaystyle d_{i} = \sum_{k=0}^n g_i(x_k)f(x_k)$.

 \item Seja uma função $f$ da qual são conhecidos os pontos $(x_0,\,f(x_0))$ e $(x_1,\,f(x_1))$. Considere que 
$L(x)$ seja o polinômio na Forma de Lagrange que interpola $f$. Além disso, considere que $N(x)$ seja o polinômio na Forma de Newton que 
interpola $f$. Prove que $L(x)$ e $N(x)$ representam um mesmo polinômio. 

\end{enumerate}
\end{document}
