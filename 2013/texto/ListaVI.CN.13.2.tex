\documentclass[12pt,a4paper]{article}
\usepackage[utf8]{inputenc}
\usepackage[brazil]{babel}
\usepackage{graphicx}
\usepackage{amssymb, amsfonts, amsmath}
\usepackage{color}
\usepackage{float}
\usepackage{enumerate}
%\usepackage{subfigure}
\usepackage[top=2.5cm, bottom=2.5cm, left=1.25cm, right=1.25cm]{geometry}

\newcommand{\sen}{\,\textrm{sen}\,}
\newcommand{\tg}{\,\textrm{tg}\,}

\begin{document}
\pagestyle{empty}

\begin{center}
\begin{tabular}{ccc}
\begin{tabular}{c}
\includegraphics[scale=0.25]{../../biblioteca/imagem/brasao-de-armas-brasil} \\
\end{tabular} & 
\begin{tabular}{c}
Ministério da Educação \\
Universidade Federal dos Vales do Jequitinhonha e Mucuri \\
Faculdade de Ciências Sociais, Aplicadas e Exatas - FACSAE \\
Departamento de Ciências Exatas - DCEX \\
Disciplina: Cálculo Numérico\\
Prof.: Luiz C. M. de Aquino\\
\end{tabular} &
\begin{tabular}{c}
\includegraphics[scale=0.25]{../../biblioteca/imagem/logo-ufvjm} \\
\end{tabular}
\end{tabular}
\end{center}

\begin{center}
 \textbf{Lista de Exercícios VI}
\end{center}

% \begin{flushleft}
%  \textbf{Observação}
% 
%  Nesta lista de exercícios, todos os sistemas de equações devem ser resolvidos por Eliminação Gaussiana.
% \end{flushleft}

\begin{enumerate}
  
  \item Resolva o sistema abaixo utilizando o Método de Eliminação Gaussiana de duas maneiras: com pivoteamento parcial e sem pivoteamento parcial. Para efetuar todas as operações considere que um dispositivo 
com quatro casas decimais de precisão foi utilizado. Além disso, considere que este dispositivo efetua o método de arredondamento usual. Compare as 
soluções obtidas usando este dispositivo com a solução exata deste sistema.

  $$%x = -1, y = 1, z = 2.
   \begin{cases}
    3x - 5y + z = -6 \\
    -x + y + 3z = 8 \\
    -7x + 3y - 6z = -2
   \end{cases}
  $$

  \item Efetue a fatoração $LU$ da matriz $A$ dada abaixo de duas maneiras: com pivoteamento parcial e sem pivoteamento parcial. 
Para efetuar todas as operações considere que um dispositivo com quatro casas decimais de precisão foi utilizado. 
Além disso, considere que este dispositivo efetua o método de arredondamento usual. Compare os resultados obtidos.

$$A =
\begin{bmatrix}
-1 & 0 & 4 \\
 2 & 1 & 1 \\
 1 & -1 &  2
\end{bmatrix}
$$
\end{enumerate}

\begin{center}
\textbf{Gabarito}
\end{center} 
\textbf{[1]} Sugestão: Comente se a solução obtida pelo Método da Eliminação Gaussiana com pivoteamento parcial foi mais próxima da solução exata. 
Compare o erro de arredondamento ao usar o pivoteamento e ao não usar.
\textbf{[2]} Sugestão: Comente se a solução obtida com o pivoteamento parcial foi mais próxima da solução exata. 
Compare o erro de arredondamento ao usar o pivoteamento e ao não usar.
\end{document}
