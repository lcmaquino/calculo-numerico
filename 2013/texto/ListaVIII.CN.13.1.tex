\documentclass[12pt,a4paper]{article}
\usepackage[utf8]{inputenc}
\usepackage[brazil]{babel}
\usepackage{graphicx}
\usepackage{amssymb, amsfonts, amsmath}
\usepackage{color}
\usepackage{float}
\usepackage{enumerate}
%\usepackage{subfigure}
\usepackage[top=2.5cm, bottom=2.5cm, left=1.25cm, right=1.25cm]{geometry}

\newcommand{\sen}{\,\textrm{sen}\,}
\newcommand{\tg}{\,\textrm{tg}\,}

\begin{document}
\pagestyle{empty}

\begin{center}
\begin{tabular}{ccc}
\begin{tabular}{c}
\includegraphics[scale=0.25]{../../biblioteca/imagem/brasao-de-armas-brasil} \\
\end{tabular} & 
\begin{tabular}{c}
Minist�rio da Educa��o \\
Universidade Federal dos Vales do Jequitinhonha e Mucuri \\
Faculdade de Ci�ncias Sociais, Aplicadas e Exatas - FACSAE \\
Departamento de Ci�ncias Exatas - DCEX \\
Disciplina: C�lculo Num�rico\\
Prof.: Luiz C. M. de Aquino\\
\end{tabular} &
\begin{tabular}{c}
\includegraphics[scale=0.25]{../../biblioteca/imagem/logo-ufvjm} \\
\end{tabular}
\end{tabular}
\end{center}

\begin{center}
 \textbf{Lista de Exerc�cios VIII}
\end{center}

\begin{enumerate}
  
  \item Determine um \textit{spline} natural c�bica que interpole os pontos da tabela abaixo:

   \begin{center}
   \begin{tabular}{c|c|c|c|c}
      $x_i$ & $-1,2$ & $-0,1$ & $0,5$ & $1,4$ \\ \hline
      $y_i$ & $-1$ & $0,2$ & $0,75$ & $-0,15$
   \end{tabular}
   \end{center}

  \item Considere uma fun��o $f$ da qual s�o conhecidos os seguintes pontos:

   \begin{center}
   \begin{tabular}{c|c|c|c|c|c|c|c|c|c|c}
      $x_i$ & $-4,2$ & $-2,8$ & $-2,2$ & $-0,75$ & $0$ & $1,2$ & $1,6$ & $3,5$ & $4$ & $5,2$\\ \hline
      $f(x_i)$ & $24$ & $15,7$ & $8,8$ & $3,6$ & $1,2$ & $0,6$ & $0,25$ & $4,4$ & $8,2$ & $15,5$
   \end{tabular}
   \end{center}

   \begin{enumerate}
    \item Fa�a um esbo�o desses pontos no plano cartesiano. A partir desse esbo�o, analise qual o grau do polin�mio que 
          parece se ajustar a estes pontos.
    \item Utilize o M�todo dos M�nimos Quadrados para determinar o polin�mio que melhor se ajusta a estes pontos (considerando o grau analisado
          no item (a)).
   \end{enumerate}
  
\end{enumerate}

\begin{center}
\textbf{Gabarito}
\end{center} 
\textbf{[1]} $S(x) = \begin{cases}
                      0,057952x^3 + 0,20863x^2 + 1,2711x + 0,32509;\,-1,2 \leq x < -0,1 \\
                      -1,1923x^3 - 0,16645x^2 + 1,2336x + 0,32384;\,-0,1 \leq x < 0,5 \\
                      0,72404x^3 - 3,0410x^2 + 2,6709x + 0,084291;\,0,5 \leq x \leq 1,4
                     \end{cases}
$
\textbf{[2]} (a) Grau $2$. (b) $\phi(x) = 1,3815x^2 - 1,9159x + 0,87686$.
\end{document}
