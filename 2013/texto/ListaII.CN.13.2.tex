\documentclass[12pt,a4paper]{article}
\usepackage[utf8]{inputenc}
\usepackage[brazil]{babel}
\usepackage{graphicx}
\usepackage{amssymb, amsfonts, amsmath}
\usepackage{color}
\usepackage{float}
\usepackage{enumerate}
%\usepackage{subfigure}
\usepackage[top=2.5cm, bottom=2.5cm, left=1.25cm, right=1.25cm]{geometry}

\newcommand{\sen}{\,\textrm{sen}\,}
\newcommand{\tg}{\,\textrm{tg}\,}

\begin{document}
\pagestyle{empty}

\begin{center}
\begin{tabular}{ccc}
\begin{tabular}{c}
\includegraphics[scale=0.25]{../../biblioteca/imagem/brasao-de-armas-brasil} \\
\end{tabular} & 
\begin{tabular}{c}
Ministério da Educação \\
Universidade Federal dos Vales do Jequitinhonha e Mucuri \\
Faculdade de Ciências Sociais, Aplicadas e Exatas - FACSAE \\
Departamento de Ciências Exatas - DCEX \\
Disciplina: Cálculo Numérico\\
Prof.: Luiz C. M. de Aquino\\
\end{tabular} &
\begin{tabular}{c}
\includegraphics[scale=0.25]{../../biblioteca/imagem/logo-ufvjm} \\
\end{tabular}
\end{tabular}
\end{center}

\begin{center}
 \textbf{Lista de Exercícios II}
\end{center}

\begin{enumerate}

  \item Explique como obter a expressão para o termo $x_n$ da sequência definida pelo Método das Cordas para uma função $f$ contínua no intervalo $[a;\,b]$ e tal que $f(a)f(b)<0$.
  
  \item Aplique o Método das Cordas para encontrar uma aproximação da raiz da função definida por $f(x) = \sen x - \dfrac{1}{5}$ no intervalo $[0;\,1]$ (considere uma tolerância de $10^{-4}$).

  \item Use o Método da Secante para encontrar a raiz aproximada da função definida por $f(x) = \cos x - \dfrac{1}{5}$ no intervalo $[1;\,2]$ (considere uma tolerância de $10^{-5}$).
  
  \item Dê exemplo de uma função contínua que possua uma única raiz no intervalo $[1; 3]$, mas para a qual não é possível aplicar o Método da Secante para aproximar essa raiz usando os chutes iniciais 
$x_0 = 1,4$ e $x_1 = 2,6$. Justifique porque não é possível usar o método no seu exemplo.
\end{enumerate}

\begin{center}
\textbf{Gabarito}
\end{center}
\textbf{[1]} Sugestão: Primeiro, determine a equação da reta que passa por $(x_n;\,f(x_n))$ e é paralela a reta passando por $(a;\,f(a))$ e $(b;\,f(b))$. Em seguida, 
defina $x_{n+1}$ como sendo a abscissa do ponto de interseção entre esta reta e o eixo $x$. 
\textbf{[2]} $x\approx 0,20133$. 
\textbf{[3]} $x\approx 1,36944$. 
\textbf{[4]} Sugestão: tente montar a função de tal modo que $f(x_0)$ seja igual a $f(x_1)$. Observação: esse exercício possui várias soluções.
\end{document}
