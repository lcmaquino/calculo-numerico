\documentclass[12pt,a4paper]{article}
\usepackage[utf8]{inputenc}
\usepackage[brazil]{babel}
\usepackage{graphicx}
\usepackage{amssymb, amsfonts, amsmath}
\usepackage{color}
\usepackage{float}
\usepackage{enumerate}
%\usepackage{subfigure}
\usepackage[top=2.5cm, bottom=2.5cm, left=1.25cm, right=1.25cm]{geometry}

\newcommand{\sen}{\,\textrm{sen}\,}
\newcommand{\tg}{\,\textrm{tg}\,}

\begin{document}
\pagestyle{empty}

\begin{center}
\begin{tabular}{ccc}
\begin{tabular}{c}
\includegraphics[scale=0.25]{../../biblioteca/imagem/brasao-de-armas-brasil} \\
\end{tabular} & 
\begin{tabular}{c}
Ministério da Educação \\
Universidade Federal dos Vales do Jequitinhonha e Mucuri \\
Faculdade de Ciências Sociais, Aplicadas e Exatas - FACSAE \\
Departamento de Ciências Exatas - DCEX \\
Disciplina: Cálculo Numérico\\
Prof.: Luiz C. M. de Aquino\\
\end{tabular} &
\begin{tabular}{c}
\includegraphics[scale=0.25]{../../biblioteca/imagem/logo-ufvjm} \\
\end{tabular}
\end{tabular}
\end{center}

\begin{center}
 \textbf{Lista de Exercícios III}
\end{center}

\begin{enumerate}
 \item Seja a função polinomial definida por $p(x) = x^4 - 2x^3 + 4x^2 - x + 5$. Utilize o Método de Horner para calcular $p(4)$ e $p'(4)$
 \item Obtenha a Sequência de Sturm para a função polinomial definida por $p(x) = -x^3 + x^2 - x + 1$.
 \item Utilize o Método de Newnton para determinar uma aproximação para a raiz da função polinomial definida por $p(x) = 2x^4 -2x^3 -22x^2 - 10x + 8$ no intervalo $[0,\,1]$ (considere uma tolerância 
 de $10^{-5}$).
 \item Considere a função polinomial definida por $p(x) = x^3 -\dfrac{7}{4}x^2 - 5x + \dfrac{35}{4}$. Sabe-se que no intervalo $[1,8;\,2,4]$ há uma raiz desta função. 
 Explique porque não podemos obter uma aproximação desta raiz utilizando o Método de Newnton com chute inicial no intervalo $(1,8;\,2)$. Discuta a viabilidade de usar o Método de Newton na resolução deste problema.
 \item Utilize um método numérico para determinar aproximadamente qual é o ponto da circunferência $x^2 + y^2 = 1$ que está mais próximo da reta $x + 2y - 4 = 0$.
 
\end{enumerate}

\begin{center}
\textbf{Gabarito}
\end{center} 
\textbf{[1]} $b_4 = 1$, $b_3 = 2$, $b_2 = 12$, $b_1 = 47$, $p(4) = b_0 = 193$. $\tilde{b}_3 = 1$, $\tilde{b}_2 = 6$, $\tilde{b}_1 = 36$, $p'(4) = \tilde{b}_0 = 191$. 
\textbf{[2]} $f_0(x) = -x^3 + x^2 - x + 1$, $f_1(x) = -3x^2 + 2x - 1$, $f_2(x) = \dfrac{4}{9}x-\dfrac{8}{9}$ e $f_3(x) = 9$.
\textbf{[3]} $x\approx 0,41421$. 
\textbf{[4]} Sugestão: note que para $x$ no intervalo $(1,8;\,2)$, temos que $p'(x) < 0$ e $p''(x) > 0$. Além disso, note que $p'(2)=0$ e para $x$ no intervalo $(2;\,2,4)$ temos que $p'(x) > 0$ e $p''(x) > 0$.
\textbf{[5]} $P\approx (0,44721; 0,89443)$. 
\end{document}
