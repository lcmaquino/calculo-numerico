\documentclass[12pt,a4paper]{article}
\usepackage[utf8]{inputenc}
\usepackage[brazil]{babel}
\usepackage{graphicx}
\usepackage{amssymb, amsfonts, amsmath}
\usepackage{color}
\usepackage{float}
\usepackage{enumerate}
%\usepackage{subfigure}
\usepackage[top=2.5cm, bottom=2.5cm, left=1.25cm, right=1.25cm]{geometry}

\newcommand{\sen}{\,\textrm{sen}\,}
\newcommand{\tg}{\,\textrm{tg}\,}

\begin{document}
\pagestyle{empty}

\begin{center}
\begin{tabular}{ccc}
\begin{tabular}{c}
\includegraphics[scale=0.25]{../../biblioteca/imagem/brasao-de-armas-brasil} \\
\end{tabular} & 
\begin{tabular}{c}
Ministério da Educação \\
Universidade Federal dos Vales do Jequitinhonha e Mucuri \\
Faculdade de Ciências Sociais, Aplicadas e Exatas - FACSAE \\
Departamento de Ciências Exatas - DCEX \\
Disciplina: Cálculo Numérico\\
Prof.: Luiz C. M. de Aquino\\
\end{tabular} &
\begin{tabular}{c}
\includegraphics[scale=0.25]{../../biblioteca/imagem/logo-ufvjm} \\
\end{tabular}
\end{tabular}
\end{center}

\begin{center}
 \textbf{Lista de Exercícios V}
\end{center}

\begin{enumerate}
  
  \item Considere o seguinte sistema de equações:
  $$%x = -1, y = 1, z = 2.
   \begin{cases}
    2x - 4y + 8z  - w = -6 \\
    -2x  - 2y + z - 7w = -5 \\
    5x  - y + z - 2w = -2 \\
    x - 4y - z + w = 8
   \end{cases}
  $$

  \begin{enumerate}
   \item Da forma como ele está arrumado, é recomendável usar diretamente o método de Gauss-Jaboci? Justifique sua resposta.
   \item Proponha uma maneira de utilizar o método de Gauss-Jacobi para obter uma solução aproximada desse sistema (considere uma tolerância de $10^{-5}$). 
         Justifique a sua proposta.
  \end{enumerate}


  \item Utilize o Método de Gauss-Seidel para resolver o mesmo sistema que você propôs no exercício 1. (b) (considere a mesma tolerância). Houve alguma vantagem em relação ao 
Método de Gauss-Jacobi?

\end{enumerate}

\begin{center}
\textbf{Gabarito}
\end{center} 
\textbf{[1]} (a) Não, pois usando o Critério das Linhas não temos a garantia da convergência. 
(b) Primeiro, podemos trocar de lugar as colunas um e três. Em seguida, podemos trocar de lugar as linhas dois e quatro. Com esta arrumação,
podemos garantir pelo Critério das Linhas que o método de Gauss-Jacobi será convergente. Utilizando então o chute inicial $z^{(0)} = -\frac{6}{8}$, 
$y^{(0)} = -2$, $x^{(0)} = -\frac{2}{5}$ e $w^{(0)} = \frac{5}{7}$, obtemos como solução $z \approx -1,3408269471$, $y \approx -1,4387214857$, 
$x\approx -0,0413311135$ e $w\approx 0,9456141095$. 
\textbf{[2]} Utilizando o mesmo chute inicial de 1. (b), obtemos $z \approx -1,3408296817$, $y \approx -1,4387216809$, 
$x\approx -0,0413335297$ e $w\approx 0,9456115342$. A vantagem em relação ao método anterior foi a redução do número de passos para atingir a 
tolerância desejada.

\end{document}
