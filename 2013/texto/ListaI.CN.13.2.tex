\documentclass[12pt,a4paper]{article}
\usepackage[utf8]{inputenc}
\usepackage[brazil]{babel}
\usepackage{graphicx}
\usepackage{amssymb, amsfonts, amsmath}
\usepackage{color}
\usepackage{float}
\usepackage{enumerate}
\usepackage{subfigure}
\usepackage[top=2.5cm, bottom=2.5cm, left=1.25cm, right=1.25cm]{geometry}

\newcommand{\sen}{\,\textrm{sen}\,}
\newcommand{\tg}{\,\textrm{tg}\,}

\begin{document}
\pagestyle{empty}

\begin{center}
\begin{tabular}{ccc}
\begin{tabular}{c}
\includegraphics[scale=0.25]{../../biblioteca/imagem/brasao-de-armas-brasil} \\
\end{tabular} & 
\begin{tabular}{c}
Ministério da Educação \\
Universidade Federal dos Vales do Jequitinhonha e Mucuri \\
Faculdade de Ciências Sociais, Aplicadas e Exatas - FACSAE \\
Departamento de Ciências Exatas - DCEX \\
Disciplina: Cálculo Numérico\\
Prof.: Luiz C. M. de Aquino\\
\end{tabular} &
\begin{tabular}{c}
\includegraphics[scale=0.25]{../../biblioteca/imagem/logo-ufvjm} \\
\end{tabular}
\end{tabular}
\end{center}

\begin{center}
 \textbf{Lista de Exercícios I}
\end{center}

\begin{enumerate}
\item Use o Método da Bisseção para encontrar uma solução aproximada das seguintes equações (considere uma tolerância de $10^{-4}$):
 \begin{enumerate}
  \item $(3x)2^x = 1$. 
  \item $\sen 2x = \ln (x - 1)$.
 \end{enumerate}

 \item Considere a função definida por $f(x) = \dfrac{2x - 3}{x-1}$. Há algum problema em aplicar o Método da Bisseção para determinar uma raiz desta função no intervalo $[0,25;\,1,25]$? Justifique sua resposta.

 \item Dado $a\in\mathbb{R}_+^*$ proponha uma maneira de usar o Método da Bisseção para calcular um valor aproximado de $\sqrt{a}$ com tolerância de $10^{-5}$. Em seguida, use a sua 
proposta para calcular o valor aproximado de $\sqrt{2}$.

 \item Dê exemplo de uma equação que envolva termos do tipo $2^u$ e $\sen u$ e cuja solução seja $x = 4$. Em seguida, determine um intervalo 
contendo $x = 4$ e considere que o Método da Bisseção será aplicado nesse intervalo. Faça uma estimativa do número de passos do método que 
serão necessários para obter a precisão de $\varepsilon = 10^{-5}$. Execute essa quantidade de passos e compare a solução aproximada com a solução 
exata da equação.

 \item Seja a função definida por $f(t) =  -\dfrac{112}{9}t^3 + \dfrac{536}{9}t^2 -\dfrac{815}{9}t + \dfrac{400}{9}$. Verifique que $\bar{u} = \dfrac{5}{4}$ é solução de $f(u) = 0$. 
Em seguida, justifique porque não é possível utilizar o Método da Bisseção para determinar uma solução aproximada de $\bar{u}$.

\end{enumerate}

\begin{center}
\textbf{Gabarito}
\end{center}
\textbf{[1]} (a) $x\approx 0,27539$. (b) $x\approx 1,7305$. 
\textbf{[2]} Sim, pois $f$ é descontínua neste intervalo. Em particular, com dois passos do método obtemos $x_2 = 1$, mas $f$ é descontínua em $x = 1$. 
\textbf{[3]} Sugestão: note que $\sqrt{a}$ é a raiz de $x^2 - a = 0$ no intervalo $[0;\,a+1]$. Observação: este exercício admite outras respostas válidas. 
\textbf{[4]} Sugestão: note que $2^0 = 1$ e $\sen\dfrac{\pi}{2} = 1$. Observação: este exercício admite várias respostas válidas.
\textbf{[5]} De fato, basta verificar que $f\left(\dfrac{5}{4}\right) = 0$. Não é possível, pois $f^\prime\left(\dfrac{5}{4}\right) = 0$ e 
$f^{\prime\prime}\left(\dfrac{5}{4}\right) > 0$.
\end{document}
