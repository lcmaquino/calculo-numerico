\documentclass[10pt,a4paper]{article}
\usepackage[utf8]{inputenc}
\usepackage[brazil]{babel}
\usepackage{graphicx}
\usepackage{amssymb, amsfonts, amsmath}
\usepackage{color}
\usepackage{float}
\usepackage{enumerate}
%\usepackage{subfigure}
\usepackage[top=2.5cm, bottom=2.5cm, left=1.25cm, right=1.25cm]{geometry}

\newcommand{\sen}{\,\textrm{sen}\,}
\newcommand{\tg}{\,\textrm{tg}\,}

\begin{document}
\pagestyle{empty}

\begin{center}
\begin{tabular}{ccc}
\begin{tabular}{c}
\includegraphics[scale=0.25]{../../biblioteca/imagem/brasao-de-armas-brasil} \\
\end{tabular} & 
\begin{tabular}{c}
Ministério da Educação \\
Universidade Federal dos Vales do Jequitinhonha e Mucuri \\
Faculdade de Ciências Sociais, Aplicadas e Exatas - FACSAE \\
Departamento de Ciências Exatas - DCEX \\
Disciplina: Cálculo Numérico\\
Prof.: Luiz C. M. de Aquino\\
\end{tabular} &
\begin{tabular}{c}
\includegraphics[scale=0.25]{../../biblioteca/imagem/logo-ufvjm} \\
\end{tabular}
\end{tabular}

Aluno(a): \rule{0.5\textwidth}{0.01cm} \quad Data: \rule{0.6cm}{0.01cm} / \rule{0.6cm}{0.01cm} / \rule{1.25cm}{0.01cm}
\end{center}

\begin{center}
 \textbf{Avaliação III}
\end{center}

\textbf{Instruções}
\begin{itemize}
 \item Todas as justificativas necessárias na solução de cada questão devem estar presentes nesta avaliação;
 \item As respostas finais de cada questão devem estar escritas de caneta;
 \item Esta avaliação tem um total de 25,0 pontos.
\end{itemize}

\begin{enumerate}
 
 \item \textbf{[4,0 pontos]} Seja uma função $f$ da qual são conhecidos os valores descritos na tabela abaixo.
  \begin{center}
   \begin{tabular}{c|c|c|c}
    $x_i$ & 1 & 1,5 & 2\\ \hline
    $f(x_i)$ & 2 & 3,125 & 6
   \end{tabular}
  \end{center}

  Determine o polinômio $p$ que interpola $f$ utilizando duas maneiras:
  \begin{enumerate}
   \item escrevendo $p$ na Forma de Lagrange;
   \item escrevendo $p$ na Forma de Newton.
  \end{enumerate}
 
 \item \textbf{[5,0 pontos]} Seja $p$ o polinômio na Forma de Lagrange que interpola os pontos $(x_0,\,y_0)$, $(x_1,\,y_1)$, \ldots, $(x_n,\,y_n)$. Vamos definir o 
polinômio $$q(x) = \prod_{i=0}^{n} (x-x_i).$$ Prove que $p$ pode ser escrito no seguinte formato: 
$$p(x) = \sum_{i=0}^n\frac{q(x)}{(x-x_i)q'(x_i)}y_i.$$

  \item \textbf{[6,0 pontos]} Sobre certa função $f$ são conhecidos os pontos $(x_k,\,f(x_k))$, com $k=0$, $1$, $2$, \ldots, $n$. Suponha que seja 
aplicado o Método dos Mínimos Quadrados para determinar a função $\phi(x) = ag_1(x) + bg_2(x)$ que melhor se ajusta a $f$. Deduza que os coeficientes $a$ e 
$b$ são a solução do sistema de equações:
$$\begin{cases}
   c_{11}a + c_{12}b = d_1 \\
   c_{21}a + c_{22}b = d_2
  \end{cases},
$$
onde $\displaystyle c_{ij} = \sum_{k=0}^n g_i(x_k)g_j(x_k)$ e $\displaystyle d_{i} = \sum_{k=0}^n g_i(x_k)f(x_k)$.

 \item \textbf{[6,0 pontos]} Considere a função definida por $g_k(x)=\sen(k\pi x)$, onde $k\in\mathbb{N}$.

   \begin{enumerate}
    \item Prove que $\displaystyle\int_{-1}^1 g_i(x)g_j(x)\,dx = \begin{cases}1,\,i = j \\ 0,\,i\neq j\end{cases}$.
    \item Utilize o Método dos Mínimos Quadrados para determinar $\displaystyle \phi(x) = \sum_{k=1}^4 a_kg_k(x)$ que melhor se ajusta a função definida por 
          $f(x) = x$ no intervalo $[-1;\, 1]$.
   \end{enumerate}

 \item \textbf{[4,0 pontos]} Utilizando o Método dos Mínimos Quadrados deseja-se determinar $\displaystyle \phi(x) = \sum_{k=0}^n a_kg_k(x)$ que melhor se ajusta 
a uma função $f$ no intervalo $[a;\, b]$. Suponha que as funções $g_0$, $g_1$, \ldots, $g_n$ sejam escolhidas de tal forma que 
$\displaystyle\int_a^b g_i(x)g_j(x)\,dx = \begin{cases}1,\,i = j \\ 0,\,i\neq j\end{cases}$. Prove que nesse caso teremos 
$a_k = \displaystyle\int_a^b f(x)g_k(x)\,dx$.

\end{enumerate}
\end{document}
