\documentclass[12pt,a4paper]{article}
\usepackage[utf8]{inputenc}
\usepackage[brazil]{babel}
\usepackage{graphicx}
\usepackage{amssymb, amsfonts, amsmath}
\usepackage{color}
\usepackage{float}
\usepackage{enumerate}
%\usepackage{subfigure}
\usepackage[top=2.5cm, bottom=2.5cm, left=1.25cm, right=1.25cm]{geometry}

\newcommand{\sen}{\,\textrm{sen}\,}
\newcommand{\tg}{\,\textrm{tg}\,}

\begin{document}
\pagestyle{empty}

\begin{center}
\begin{tabular}{ccc}
\begin{tabular}{c}
\includegraphics[scale=0.25]{../../biblioteca/imagem/brasao-de-armas-brasil} \\
\end{tabular} & 
\begin{tabular}{c}
Ministério da Educação \\
Universidade Federal dos Vales do Jequitinhonha e Mucuri \\
Faculdade de Ciências Sociais, Aplicadas e Exatas - FACSAE \\
Departamento de Ciências Exatas - DCEX \\
Disciplina: Cálculo Numérico\\
Prof.: Luiz C. M. de Aquino\\
\end{tabular} &
\begin{tabular}{c}
\includegraphics[scale=0.25]{../../biblioteca/imagem/logo-ufvjm} \\
\end{tabular}
\end{tabular}
\end{center}

\begin{center}
 \textbf{Lista de Exercícios X}
\end{center}

\begin{enumerate}
  
  \item Considere uma função $f$ da qual são conhecidos os seguintes pontos:

   \begin{center}
   \begin{tabular}{c|c|c|c|c|c|c|c|c|c}
      $x_i$ & $2,0$ & $2,375$ & $2,75$ & $3,125$ & $3,5$ & $3,875$ & $4,25$ & $4,625$ & $5$\\ \hline
      $f(x_i)$ & $0,4134$ & $0,2026$ & $0,053$ & $0,0001$ & $0,0352$ & $0,1156$ & $0,1885$ & $0,2146$ & $0,1839$
   \end{tabular}
   \end{center}

   Calcule o valor aproximado de $\displaystyle \int_2^5 f(x)\,dx$ usando:
   \begin{enumerate}
    \item a Regra do Trapézio;
    \item a Regra 1/3 de Simpson.
   \end{enumerate}
  
   \item Considere a elipse $\dfrac{x^2}{4} + \dfrac{y^2}{9} = 1$. Proponha uma maneira de utilizar a Regra 1/3 de Simpson 
para calcular o valor aproximado de sua área. Em seguida, compare esta aproximação com o valor exato de sua área.

   \item Aplique a Regra do Trapézio para aproximar $\displaystyle\int_0^2 1 + \sen 4\pi x\,dx$ considerando que o intervalo $[0, 2]$ seja 
         divido em $8$ partes iguais. Discuta o resultado obtido comparando com o valor exato da integral.

   \item Considere que o intervalo $[a,\,b]$ foi dividido em $n$ partes iguais, obtendo-se os valores $x_i = a + hi$, com $h = \dfrac{b-a}{n}$ e 
   $i=0$, $1$, $2$, \ldots, $n$. Suponha que uma função $f$ (contínua em $[a,\,b]$) foi aproximada em cada subintervalo $[x_i,\,x_{i+1}]$ pela função constante $c_i(x) = f(x_i)$.
   Nestas condições, obtenha a expressão para a aproximação de $\displaystyle\int_a^b f(x)\,dx$. Interprete geometricamente esta aproximação.

\end{enumerate}

\begin{center}
\textbf{Gabarito}
\end{center} 
\textbf{[1]} (a) $0,41559375$. (b) $0,4102875$. 
\textbf{[2]} Sugestão: note que a área desta elipse pode ser obtida por $\displaystyle 4\int_0^2 \sqrt{9\left(1 - \dfrac{x^2}{4}\right)}\,dx$, que é 
igual a $6\pi$. 
\textbf{[3]} Aplicando a Regra do Trapézio, obtemos $2$. Esta aproximação na verdade coincide com o valor exato da integral! Note que 
$\displaystyle \int_{\frac{k}{2}}^{\frac{k+1}{2}} 1 + \sen 4\pi x\,dx$ (com $k = 0$, $1$, $2$, $3$) coincide com a área de um retângulo de base $0,5$ e altura $1$. 
Por outro lado, aplicando a Regra do Trapézio no intervalo $\left[\frac{k}{2},\,\frac{k+1}{2}\right]$, obtemos exatamente um retângulo de base $0,5$ e altura $1$. 
\textbf{[4]} Teremos $\displaystyle\int_a^b f(x)\,dx \approx h\sum_{i=0}^{n-1}f(x_i)$. Geometricamente, esta aproximação representa a soma das áreas dos retângulos de base $[x_i,\,x_{i+1}]$ e altura $f(x_i)$. 
\end{document}
