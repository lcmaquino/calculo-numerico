\documentclass[12pt,a4paper]{article}
\usepackage[latin1]{inputenc}
\usepackage[brazil]{babel}
\usepackage{graphicx}
\usepackage{amssymb, amsfonts, amsmath}
\usepackage{color}
\usepackage{float}
\usepackage{enumerate}
%\usepackage{subfigure}
\usepackage[top=2.5cm, bottom=2.5cm, left=1.25cm, right=1.25cm]{geometry}

\newcommand{\sen}{\,\textrm{sen}\,}
\newcommand{\tg}{\,\textrm{tg}\,}

\begin{document}
\pagestyle{empty}

\begin{center}
\begin{tabular}{ccc}
\begin{tabular}{c}
\includegraphics[scale=0.25]{../../biblioteca/imagem/brasao-de-armas-brasil} \\
\end{tabular} & 
\begin{tabular}{c}
Ministério da Educação \\
Universidade Federal dos Vales do Jequitinhonha e Mucuri \\
Faculdade de Ciências Sociais, Aplicadas e Exatas - FACSAE \\
Departamento de Ciências Exatas - DCEX \\
Disciplina: Cálculo Numérico\\
Prof.: Luiz C. M. de Aquino\\
\end{tabular} &
\begin{tabular}{c}
\includegraphics[scale=0.25]{../../biblioteca/imagem/logo-ufvjm} \\
\end{tabular}
\end{tabular}
\end{center}

\begin{center}
 \textbf{Lista de Exercícios V}
\end{center}

% \begin{flushleft}
%  \textbf{Observação}
% 
%  Nesta lista de exercícios, todos os sistemas de equações devem ser resolvidos por Eliminação Gaussiana.
% \end{flushleft}

\begin{enumerate}
  
  \item Resolva o sistema abaixo utilizando o Método de Eliminação Gaussiana de duas maneiras: com pivoteamento parcial e sem pivoteamento parcial. Para efetuar todas as operações considere que um dispositivo 
com quatro casas decimais de precisão foi utilizado. Além disso, considere que este dispositivo efetua o método de arredondamento usual. Compare as 
soluções obtidas usando este dispositivo com a solução exata deste sistema.

  $$%x = -1, y = 1, z = 2.
   \begin{cases}
    3x - 5y + z = -6 \\
    -x + y + 3z = 8 \\
    -7x + 3y - 6z = -2
   \end{cases}
  $$

  \item Pesquise uma maneira de utilizar o Método de Eliminação Gaussiana para calcular o determinante de uma matriz. Em seguida, use este método para calcular 
o determinante da matriz:
$$A=\begin{bmatrix}3 & - 5 & 1 \\ -1 & 1 & 3 \\ -7 & 3 & -6\end{bmatrix}.$$

  \item Considere as matrizes:
$$A=\begin{bmatrix}1 & - 1 & 2 \\ 3 & 7 & -4 \\ 5 & -2 & 1\end{bmatrix}, \, 
x=\begin{bmatrix}x_1 \\ x_2 \\ x_3\end{bmatrix}, \, 
b=\begin{bmatrix}-2 \\ 1 \\ 5\end{bmatrix} \textrm{ e }\,
c=\begin{bmatrix}1 \\ -1 \\ 3\end{bmatrix}. 
$$

  \begin{enumerate}
    \item Determine a fatoração $LU$ de $A$.
    \item Use a fatoração do item anterior para resolver os sistemas $Ax = b$ e $Ax = c$.
  \end{enumerate}

  \item Suponha que uma matriz $A$ foi fatorada no formato $LU$. Preencha os espaços em branco abaixo de modo a determinar 
as matrizes $A$, $L$ e $U$.

$$
\begin{bmatrix}\square & \square & \square \\ 1 & 0 & \square \\ \square & -4 & -3\end{bmatrix} = 
\begin{bmatrix}1 & 0 & 0 \\ \square & 1 & 0 \\ -5 & \square & 1\end{bmatrix}
\begin{bmatrix}-2 & \square & 4 \\ 0 & \square & 3 \\ 0 & 0 & -1\end{bmatrix}
$$

\end{enumerate}

\begin{center}
\textbf{Gabarito}
\end{center} 
\textbf{[1]} A solução obtida pelo Método da Eliminação Gaussiana com pivoteamento parcial foi mais próxima da solução exata. 
\textbf{[2]} $\det A = 94$.
\textbf{[3]} (a) $L = \begin{bmatrix}1 & 0 & 0 \\ 3 & 1 & 0 \\ 5 & \frac{3}{10} & 1\end{bmatrix}$ e 
$U = \begin{bmatrix} 1 & -1 & 2 \\ 0 & 10 & -10 \\ 0 & 0 & -6 \end{bmatrix}$. 
(b) $x = 
\begin{bmatrix}
\frac{17}{20} \\
-\frac{29}{20} \\
-\frac{43}{20}
\end{bmatrix}
$. 
$x = 
\begin{bmatrix}
\frac{7}{15} \\
-\frac{4}{15} \\
\frac{2}{15}
\end{bmatrix}
$. 
\textbf{[4]} $A = \begin{bmatrix}-2 & 2 & 4 \\ 1 & 0 & 1\\ 10 & -4 & -3\end{bmatrix}$, 
$L = \begin{bmatrix}1 & 0 & 0 \\ -\frac{1}{2} & 1 & 0 \\ -5 & 6 & 1\end{bmatrix}$ e  
$U = \begin{bmatrix}-2 & 2 & 4 \\ 0 & 1 & 3 \\ 0 & 0 & -1\end{bmatrix}$
\end{document}
