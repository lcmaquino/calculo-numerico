\documentclass[12pt,a4paper]{article}
\usepackage[utf8]{inputenc}
\usepackage[brazil]{babel}
\usepackage{graphicx}
\usepackage{amssymb, amsfonts, amsmath}
\usepackage{color}
\usepackage{float}
\usepackage{enumerate}
%\usepackage{subfigure}
\usepackage[top=2.5cm, bottom=2.5cm, left=1.25cm, right=1.25cm]{geometry}

\newcommand{\sen}{\,\textrm{sen}\,}
\newcommand{\tg}{\,\textrm{tg}\,}

\begin{document}
\pagestyle{empty}

\begin{center}
\begin{tabular}{ccc}
\begin{tabular}{c}
\includegraphics[scale=0.25]{../../biblioteca/imagem/brasao-de-armas-brasil} \\
\end{tabular} & 
\begin{tabular}{c}
Ministério da Educação \\
Universidade Federal dos Vales do Jequitinhonha e Mucuri \\
Faculdade de Ciências Sociais, Aplicadas e Exatas - FACSAE \\
Departamento de Ciências Exatas - DCEX \\
Disciplina: Cálculo Numérico\\
Prof.: Luiz C. M. de Aquino\\
\end{tabular} &
\begin{tabular}{c}
\includegraphics[scale=0.25]{../../biblioteca/imagem/logo-ufvjm} \\
\end{tabular}
\end{tabular}

Aluno(a): \rule{0.5\textwidth}{0.01cm} \quad Data: \rule{0.6cm}{0.01cm} / \rule{0.6cm}{0.01cm} / \rule{1.25cm}{0.01cm}
\end{center}

\begin{center}
 \textbf{Avaliação I}
\end{center}

\textbf{Instruções}
\begin{itemize}
 \item Todas as justificativas necessárias na solução de cada questão devem estar presentes nesta avaliação;
 \item As respostas finais de cada questão devem estar escritas de caneta;
 \item Esta avaliação tem um total de 25,0 pontos.
\end{itemize}

\begin{enumerate}
 \item \textbf{[6,0 pontos]} Dado $a\in\mathbb{R}_+^*$ proponha uma maneira de usar o Método da Bisseção para calcular um valor aproximado de $\sqrt{a}$ com tolerância de $10^{-5}$. Em seguida, use a sua 
proposta para calcular o valor aproximado de $\sqrt{2}$.

 \item \textbf{[4,0 pontos]} Dê exemplo de uma função contínua que possua uma única raiz no intervalo $[1; 3]$, mas para a qual não é possível 
aplicar o Método da Secante para aproximar essa raiz usando os chutes iniciais 
$x_0 = 1,4$ e $x_1 = 2,6$. Justifique porque não é possível usar o método no seu exemplo.
 
 \item \textbf{[5,0 pontos]} A cada passo no Método da Falsa Posição, escolhemos $x_k = \dfrac{a_k|f(b_k)| + b_k|f(a_k)|}{|f(a_k)|+|f(b_k)|}$, sendo que no intervalo $[a_k;\,b_k]$ temos $f(a_k)f(b_k)<0$. Prove que 
 esta escolha de $x_k$ coincide com a abscissa do ponto de interseção entre o eixo $x$ e a reta passando por $(a_k,\,f(a_k))$ e $(b_k,\,f(b_k))$.

 \item \textbf{[4,0 pontos]} Explique como utilizar o Método de Newton para determinar aproximadamente qual é o ponto da circunferência 
$x^2 + y^2 = 1$ que está mais próximo da reta $x + 2y - 4 = 0$.

 \item \textbf{[6,0 pontos]} Seja $f:[0;\, 1]\to [0;\,1]$ uma função contínua em todo o seu domínio. Prove que para todo $n\in\mathbb{N}^*$, existe $\bar{x}\in[0;\,1]$ que é 
 solução da equação $f(x) - x^n = 0$. Indique qual é a interpretação geométrica dessa afirmação.

\end{enumerate}
\end{document}
