\documentclass[12pt,a4paper]{article}
\usepackage[utf8]{inputenc}
\usepackage[brazil]{babel}
\usepackage{graphicx}
\usepackage{amssymb, amsfonts, amsmath}
\usepackage{color}
\usepackage{float}
\usepackage{enumerate}
%\usepackage{subfigure}
\usepackage[top=2.5cm, bottom=2.5cm, left=1.25cm, right=1.25cm]{geometry}

\newcommand{\sen}{\,\textrm{sen}\,}
\newcommand{\tg}{\,\textrm{tg}\,}

\begin{document}
\pagestyle{empty}

\begin{center}
\begin{tabular}{ccc}
\begin{tabular}{c}
\includegraphics[scale=0.25]{../../biblioteca/imagem/brasao-de-armas-brasil} \\
\end{tabular} & 
\begin{tabular}{c}
Ministério da Educação \\
Universidade Federal dos Vales do Jequitinhonha e Mucuri \\
Faculdade de Ciências Sociais, Aplicadas e Exatas - FACSAE \\
Departamento de Ciências Exatas - DCEX \\
Disciplina: Cálculo Numérico\\
Prof. Me. Luiz C. M. de Aquino\\
\end{tabular} &
\begin{tabular}{c}
\includegraphics[scale=0.25]{../../biblioteca/imagem/logo-ufvjm} \\
\end{tabular}
\end{tabular}

Aluno(a): \rule{0.5\textwidth}{0.01cm} \quad Data: \rule{0.6cm}{0.01cm} / \rule{0.6cm}{0.01cm} / \rule{1.25cm}{0.01cm}
\end{center}

\begin{center}
 \textbf{Avaliação II}
\end{center}

\textbf{Instruções}
\begin{itemize}
 \item Todas as justificativas necessárias na solução de cada questão devem estar presentes nesta avaliação;
 \item As respostas finais de cada questão devem estar escritas de caneta;
 \item Esta avaliação tem um total de 25,0 pontos.
\end{itemize}

\begin{enumerate}
 \item \textbf{[5,0 pontos]} Um fabricante de móveis produz cadeiras, mesinhas de centro e mesas
de jantar. Cada cadeira leva 10 minutos para ser lixada, 6 minutos
para ser tingida e 12 minutos para ser envernizada. Cada mesinha de
centro leva 12 minutos para ser lixada, 8 minutos para ser tingida
e 12 minutos para ser envernizada. Cada mesa de jantar leva 15 minutos
para ser lixada, 12 minutos para ser tingida e 18 minutos para ser
envernizada. A bancada para lixar fica disponível 1.340 minutos por
semana, a bancada para tingir 940 minutos por semana e a bancada para
envernizar 1.560 minutos por semana. Quantos móveis devem ser fabricados
(por semana) de cada tipo para que as bancadas sejam plenamente utilizadas? (Observação: o exercício deve ser resolvido 
utilizando o Método de Eliminação Gaussiana).

 \item \textbf{[6,0 pontos]} Considere as matrizes:
$$A=\begin{bmatrix}1 & - 1 & 2 \\ 3 & 7 & -4 \\ 5 & -2 & 1\end{bmatrix}; \, 
x=\begin{bmatrix}x_1 \\ x_2 \\ x_3\end{bmatrix}; \,
b=\begin{bmatrix}-2 \\ 1 \\ 5\end{bmatrix}. 
$$

  \begin{enumerate}
    \item Determine a fatoração $LU$ de $A$.
    \item Explique como usar a fatoração do item anterior para resolver o sistema $Ax = b$ (atenção: não é necessário resolver o sistema).
  \end{enumerate}

 \item \textbf{[4,0 pontos]} Considere o seguinte sistema de equações:
  $$%x = -1, y = 1, z = 2.
   \begin{cases}
    2x - 4y + 8z  - w = -6 \\
    -2x  - 2y + z - 7w = -5 \\
    5x  - y + z - 2w = -2 \\
    x - 4y - z + w = 8
   \end{cases}
  $$

  \begin{enumerate}
   \item Da forma como ele está arrumado, é recomendável usar diretamente o método de Gauss-Jacobi? Justifique sua resposta.
   \item Arrume esse sistema de modo a utilizar o método de Gauss-Jacobi. Justifique sua arrumação. Em seguida, exiba as equações 
         utilizadas pelo método para obter uma solução aproximada desse sistema.
  \end{enumerate}

 \item \textbf{[6,0 pontos]} Considere um sistema de equações lineares dado por:
$$\begin{cases}
   a_{11}x_1 + a_{12}x_2 + a_{13}x_3 = b_1 \\
   a_{21}x_1 + a_{22}x_2 + a_{23}x_3 = b_2 \\
   a_{31}x_1 + a_{32}x_2 + a_{33}x_3 = b_3
  \end{cases}
$$

Suponha que é possível aplicar o Método de Gauss-Seidel nesse sistema com garantia de convergência. Determine as matrizes 
$$x^{(k)} = \begin{bmatrix} x^{(k)}_1 \\ x^{(k)}_2 \\ x^{(k)}_3\end{bmatrix}, \,B,\, C \textrm{ e } d,$$ 
de tal modo a escrever este método no formato $$Bx^{(k+1)} = Cx^{(k)} + d.$$
%xk+1 = 1/a11(b1 - a12yk - a13zk)
%yk+1 = 1/a22(b2 - a21x(k+1) - a23zk)
%zk+1 = 1/a33(b3 - a31x(k+1) - a32y(k+1))

 \item \textbf{[4,0 pontos]} Suponha que a fatoração $LU$ de uma matriz $A$ seja conhecida. Explique como usar essa informação para resolver o 
sistema $A^Tx = b$.

\end{enumerate}
\end{document}
