\documentclass[12pt,a4paper]{article}
\usepackage[utf8]{inputenc}
\usepackage[brazil]{babel}
\usepackage{graphicx}
\usepackage{amssymb, amsfonts, amsmath}
\usepackage{color}
\usepackage{float}
\usepackage{enumerate}
%\usepackage{subfigure}
\usepackage[top=2.5cm, bottom=2.5cm, left=1.25cm, right=1.25cm]{geometry}

\DeclareMathOperator{\sen}{sen}
\DeclareMathOperator{\tg}{tg}

\begin{document}
\pagestyle{empty}

\begin{center}
\begin{tabular}{ccc}
\begin{tabular}{c}
\includegraphics[scale=0.25]{../../biblioteca/imagem/brasao-de-armas-brasil} \\
\end{tabular} & 
\begin{tabular}{c}
Ministério da Educação \\
Universidade Federal dos Vales do Jequitinhonha e Mucuri \\
Faculdade de Ciências Sociais, Aplicadas e Exatas - FACSAE \\
Departamento de Ciências Exatas - DCEX \\
Disciplina: Cálculo Numérico\\
Prof. Me. Luiz C. M. de Aquino\\
\end{tabular} &
\begin{tabular}{c}
\includegraphics[scale=0.25]{../../biblioteca/imagem/logo-ufvjm} \\
\end{tabular}
\end{tabular}

Aluno(a): \rule{0.5\textwidth}{0.01cm} \quad Data: \rule{0.6cm}{0.01cm} / \rule{0.6cm}{0.01cm} / \rule{1.25cm}{0.01cm}
\end{center}

\begin{center}
 \textbf{Avaliação I}
\end{center}

\textbf{Instruções}
\begin{itemize}
 \item Todas as justificativas necessárias na solução de cada questão devem estar presentes nesta avaliação;
 \item As respostas finais de cada questão devem estar escritas de caneta;
 \item Esta avaliação tem um total de 25,0 pontos.
\end{itemize}

\begin{enumerate}
 \item \textbf{[6,0 pontos]} Utilize os conhecimentos de Cálculo para provar que os gráficos das funções definidas por 
$f(x) = \cos\left(x^2\right)$ e $g(x) = x^3$ possuem um único ponto de interseção. Em seguida, 
explique como utilizar o Método da Bisseção para determinar de modo aproximado esse ponto.

 \item \textbf{[4,0 pontos]} Seja a função definida por $f(t) =  -\dfrac{112}{9}t^3 + \dfrac{536}{9}t^2 -\dfrac{815}{9}t + \dfrac{400}{9}$. Verifique que $\bar{t} = \dfrac{5}{4}$ é solução de $f(t) = 0$. 
Em seguida, justifique porque não é possível utilizar o Método da Bisseção para determinar uma solução aproximada de $\bar{t}$.
 
 \item \textbf{[5,0 pontos]} Explique como obter a expressão para o termo $x_n$ da sequência definida pelo Método das Cordas para uma função $f$ contínua no intervalo $[a;\,b]$ e tal que $f(a)f(b)<0$.

 \item \textbf{[4,0 pontos]} Explique como utilizar o Método de Newton para determinar de modo aproximado o valor máximo da função definida por $f(x) = \cos x + \sen x$ no intervalo $[0; 1]$.
 
 \item \textbf{[6,0 pontos]} Crie um exemplo de equação cuja solução seja $x = \sqrt{2}$ e de tal modo que essa equação envolva termos exponeciais e termos trigonométricos. 
 Apresente uma interpretação geométrica para solução dessa equação. Por fim, explique como utilizar o Método de Newton para determinar uma solução aproximada dessa equação.

\end{enumerate}
\end{document}
