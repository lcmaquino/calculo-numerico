\documentclass[12pt,a4paper]{article}
\usepackage[utf8]{inputenc}
\usepackage[brazil]{babel}
\usepackage{graphicx}
\usepackage{amssymb, amsfonts, amsmath}
\usepackage{color}
\usepackage{float}
\usepackage{enumerate}
\usepackage{subfigure}
\usepackage[top=2.5cm, bottom=2.5cm, left=1.25cm, right=1.25cm]{geometry}

\DeclareMathOperator{\sen}{sen}
\DeclareMathOperator{\tg}{tg}
\DeclareMathOperator{\arcsen}{arcsen}
\DeclareMathOperator{\arc}{arctg}

\newcommand{\limite}[3]{\displaystyle\lim_{#2 \to #3} #1}
\newcommand{\integral}[4]{\displaystyle\int_{#3}^{#4} #1\,d#2}

\begin{document}
\pagestyle{empty}

\begin{center}
\begin{tabular}{ccc}
\begin{tabular}{c}
\includegraphics[scale=0.25]{../../biblioteca/imagem/brasao-de-armas-brasil} \\
\end{tabular} & 
\begin{tabular}{c}
Ministério da Educação \\
Universidade Federal dos Vales do Jequitinhonha e Mucuri \\
Faculdade de Ciências Sociais, Aplicadas e Exatas - FACSAE \\
Departamento de Ciências Exatas - DCEX \\
Disciplina: Cálculo Numérico\\
Prof. Me. Luiz C. M. de Aquino\\
\end{tabular} &
\begin{tabular}{c}
\includegraphics[scale=0.25]{../../biblioteca/imagem/logo-ufvjm} \\
\end{tabular}
\end{tabular}
\end{center}

\begin{center}
 \textbf{Lista de Exercícios II}
\end{center}

\begin{enumerate}
  \item Explique como obter a expressão para o termo $x_n$ da sequência definida pelo Método das Cordas para uma função $f$ contínua no intervalo $[a;\,b]$ e tal que $f(a)f(b)<0$.
  
  \item Considere a função definida por $f(x) = \sen x - \dfrac{1}{5}$. Aplique o Método da Bisseção para 
determinar uma aproximação da raiz de $f$ no intervalo $[0;\,1]$, com tolerância de $10^{-4}$. Em seguida, 
aplique o Método das Cordas para também encontrar uma aproximação dessa raiz, considerando o mesmo intervalo e 
tolerância e usando $x_0=0,5$ como chute inicial. Comparando os dois métodos, houve alguma vantagem 
em usar o Método das Cordas?

\end{enumerate}

\begin{center}
\textbf{Gabarito}
\end{center}
\textbf{[1]} Sugestão: Primeiro, determine a equação da reta que passa por $(x_n;\,f(x_n))$ e é paralela a reta passando por $(a;\,f(a))$ e $(b;\,f(b))$. Em seguida, 
defina $x_{n+1}$ como sendo a abscissa do ponto de interseção entre esta reta e o eixo $x$. 
\textbf{[2]} Método da bisseção: $x\approx 0,201416015625$. Método das Cordas: $x\approx 0,201333044061041$. 
Comparando os métodos, a vantagem de usar o Método das Cordas foi executar menos passos para obter a aproximação desejada. 
\end{document}
