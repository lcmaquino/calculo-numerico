\documentclass[12pt,a4paper]{article}
\usepackage[utf8]{inputenc}
\usepackage[brazil]{babel}
\usepackage{graphicx}
\usepackage{amssymb, amsfonts, amsmath}
\usepackage{color}
\usepackage{float}
\usepackage{enumerate}
\usepackage{subfigure}
\usepackage[top=2.5cm, bottom=2.5cm, left=1.25cm, right=1.25cm]{geometry}

\DeclareMathOperator{\sen}{sen}
\DeclareMathOperator{\tg}{tg}
\DeclareMathOperator{\arcsen}{arcsen}
\DeclareMathOperator{\arc}{arctg}

\newcommand{\limite}[3]{\displaystyle\lim_{#2 \to #3} #1}
\newcommand{\integral}[4]{\displaystyle\int_{#3}^{#4} #1\,d#2}

\begin{document}
\pagestyle{empty}

\begin{center}
\begin{tabular}{ccc}
\begin{tabular}{c}
\includegraphics[scale=0.25]{../../biblioteca/imagem/brasao-de-armas-brasil} \\
\end{tabular} & 
\begin{tabular}{c}
Ministério da Educação \\
Universidade Federal dos Vales do Jequitinhonha e Mucuri \\
Faculdade de Ciências Sociais, Aplicadas e Exatas - FACSAE \\
Departamento de Ciências Exatas - DCEX \\
Disciplina: Cálculo Numérico\\
Prof. Me. Luiz C. M. de Aquino\\
\end{tabular} &
\begin{tabular}{c}
\includegraphics[scale=0.25]{../../biblioteca/imagem/logo-ufvjm} \\
\end{tabular}
\end{tabular}
\end{center}

\begin{center}
 \textbf{Lista de Exercícios III}
\end{center}

\begin{enumerate}
  \item Use o Método da Secante para encontrar a raiz aproximada da função definida por $f(x) = \cos x - \dfrac{1}{5}$ no intervalo $[1;\,2]$ (considere uma tolerância de $10^{-5}$).
  \item Use o Método da Falsa Posição para encontrar a raiz aproximada da equação $e^{x} - e^{-x} = 2\cos x$ no intervalo $[0;\,1]$ (considere uma tolerância de $10^{-5}$).
  \item Dê exemplo de uma função contínua que possua uma única raiz no intervalo $[1; 3]$, mas para a qual 
não é possível aplicar o Método da Secante para aproximar essa raiz usando os chutes iniciais 
$x_0 = 1,4$ e $x_1 = 2,6$. Justifique porque não é possível usar o método no seu exemplo.
  \item A cada passo no Método da Falsa Posição, escolhemos $x_k = \dfrac{a_k|f(b_k)| + b_k|f(a_k)|}{|f(b_k)|+|f(a_k)|}$, sendo que no intervalo $[a_k;\,b_k]$ temos $f(a_k)f(b_k)<0$. Prove que 
 esta escolha de $x_k$ coincide com a abscissa do ponto de interseção entre o eixo $x$ e a reta passando por $(a_k,\,f(a_k))$ e $(b_k,\,f(b_k))$.

\end{enumerate}

\begin{center}
\textbf{Gabarito}
\end{center}
\textbf{[1]} $x\approx 1,36943838143575$ (Observação: a sua resposta pode ser um pouco diferente, porém próxima dessa. Isso depende do seu chute inicial). 
\textbf{[2]} $x\approx 0,703289145174550$. 
\textbf{[3]} Sugestão: tente montar a função de tal modo que $f(x_0)$ seja igual a $f(x_1)$. Observação: esse exercício possui várias soluções. 
\textbf{[4]} Sugestão: Primeiro, como $f(a_k)f(b_k)<0$, deduza que $x_k = \dfrac{a_k|f(b_k)| + b_k|f(a_k)|}{|f(b_k)|+|f(a_k)|} = \dfrac{a_kf(b_k) - b_kf(a_k)}{f(b_k)-f(a_k)}$. 
Em seguida, determine a equação da reta que passa por $(a_k,\,f(a_k))$ e $(b_k,\,f(b_k))$. Por fim, 
determine a abscissa do ponto de interseção entre esta reta e o eixo $x$. Compare esta abscissa com $x_k$. 
\end{document}
