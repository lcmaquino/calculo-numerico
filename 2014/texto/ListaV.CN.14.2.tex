\documentclass[12pt,a4paper]{article}
\usepackage[latin1]{inputenc}
\usepackage[brazil]{babel}
\usepackage{graphicx}
\usepackage{amssymb, amsfonts, amsmath}
\usepackage{color}
\usepackage{float}
\usepackage{enumerate}
%\usepackage{subfigure}
\usepackage[top=2.5cm, bottom=2.5cm, left=1.25cm, right=1.25cm]{geometry}

\newcommand{\sen}{\,\textrm{sen}\,}
\newcommand{\tg}{\,\textrm{tg}\,}

\begin{document}
\pagestyle{empty}

\begin{center}
\begin{tabular}{ccc}
\begin{tabular}{c}
\includegraphics[scale=0.25]{../../biblioteca/imagem/brasao-de-armas-brasil} \\
\end{tabular} & 
\begin{tabular}{c}
Ministério da Educação \\
Universidade Federal dos Vales do Jequitinhonha e Mucuri \\
Faculdade de Ciências Sociais, Aplicadas e Exatas - FACSAE \\
Departamento de Ciências Exatas - DCEX \\
Disciplina: Cálculo Numérico\\
Prof.: Luiz C. M. de Aquino\\
\end{tabular} &
\begin{tabular}{c}
\includegraphics[scale=0.25]{../../biblioteca/imagem/logo-ufvjm} \\
\end{tabular}
\end{tabular}
\end{center}

\begin{center}
 \textbf{Lista de Exercícios V}
\end{center}

\begin{enumerate}
  
  \item Seja uma função $f$ da qual são conhecidos os valores descritos na tabela abaixo.
%p(x) = x^3 - x^2 + 2
\begin{center}
 \begin{tabular}{c|c|c|c|c}
  $x_i$ & 1 & 1,5 & 2 & 2,5\\ \hline
  $f(x_i)$ & 2 & 3,125 & 6 & 11,375
 \end{tabular}
\end{center}

  Determine o polinômio $p$ que interpola $f$ utilizando três maneiras:
  \begin{enumerate}
   \item resolvendo o sistema formado pelas equações $p(x_i) = f(x_i)$;
   \item escrevendo $p$ na Forma de Lagrange;
   \item escrevendo $p$ na Forma de Newton.
  \end{enumerate}

  \item Seja $p$ o polinômio na Forma de Lagrange que interpola os pontos $(x_0,\,y_0)$, $(x_1,\,y_1)$, \ldots, $(x_n,\,y_n)$. Vamos definir o 
polinômio $$q(x) = \prod_{i=0}^{n} (x-x_i).$$ Prove que $p$ pode ser escrito no seguinte formato: 
$$p(x) = \sum_{i=0}^n\frac{q(x)}{(x-x_i)q'(x_i)}y_i.$$

  \item Seja uma função $f$ da qual são conhecidos os pontos $(x_0,\,f(x_0))$ e $(x_1,\,f(x_1))$. Considere que 
$L(x)$ seja o polinômio na Forma de Lagrange que interpola $f$. Além disso, considere que $N(x)$ seja o polinômio na Forma de Newton que 
interpola $f$. Prove que $L(x)$ e $N(x)$ representam um mesmo polinômio.

  \item Considere uma função $f$ da qual são conhecidos os seguintes pontos:

   \begin{center}
   \begin{tabular}{c|c|c|c|c|c|c|c|c|c|c}
      $x_i$ & $-4,2$ & $-2,8$ & $-2,2$ & $-0,75$ & $0$ & $1,2$ & $1,6$ & $3,5$ & $4$ & $5,2$\\ \hline
      $f(x_i)$ & $24$ & $15,7$ & $8,8$ & $3,6$ & $1,2$ & $0,6$ & $0,25$ & $4,4$ & $8,2$ & $15,5$
   \end{tabular}
   \end{center}

   \begin{enumerate}
    \item Faça um esboço desses pontos no plano cartesiano. A partir desse esboço, analise qual o grau do polinômio que 
          parece se ajustar a estes pontos.
    \item Utilize o Método dos Mínimos Quadrados para determinar o polinômio que melhor se ajusta a estes pontos (considerando o grau analisado
          no item (a)).
   \end{enumerate}

   \item Sobre certa função $f$ são conhecidos os pontos $(x_k,\,f(x_k))$, com $k=0$, $1$, $2$, \ldots, $n$. Suponha que seja 
aplicado o Método dos Mínimos Quadrados para determinar a função $\phi(x) = ag_1(x) + bg_2(x)$ que melhor se ajusta a $f$. Deduza que os coeficientes $a$ e 
$b$ são a solução do sistema de equações:
$$\begin{cases}
   c_{11}a + c_{12}b = d_1 \\
   c_{21}a + c_{22}b = d_2
  \end{cases},
$$
onde $\displaystyle c_{ij} = \sum_{k=0}^n g_i(x_k)g_j(x_k)$ e $\displaystyle d_{i} = \sum_{k=0}^n g_i(x_k)f(x_k)$.

   \item Considere os polinômios:
   $$p_0(x) = 1;\, p_1(x) = x;\,p_2(x) = \frac{1}{2}(3x^2 - 1).$$

   \begin{enumerate}
    \item Verifique que $\displaystyle\int_{-1}^1 p_i(x)p_j(x)\,dx = 0$, sempre que $i\neq j$.
    \item Utilize o Método dos Mínimos Quadrados para determinar $\phi(x) = a_0p_0(x) + a_1p_1(x) + a_2p_2(x)$ que melhor se ajusta a função definida por 
          $f(x) = \left(x - \dfrac{1}{2}\right)^4$ no intervalo $[-1,\, 1]$.
   \end{enumerate}

 \item Considere a função definida por $g_k(x)=\sen(k\pi x)$, onde $k\in\mathbb{N}^*$.

   \begin{enumerate}
    \item Prove que $\displaystyle\int_{-1}^1 g_i(x)g_j(x)\,dx = \begin{cases}1,\,i = j \\ 0,\,i\neq j\end{cases}$.
    \item Utilize o Método dos Mínimos Quadrados para determinar $\displaystyle \phi(x) = \sum_{k=1}^4 a_kg_k(x)$ que melhor se ajusta a função definida por 
          $f(x) = x$ no intervalo $[-1;\, 1]$.
   \end{enumerate}

     \item Utilizando o Método dos Mínimos Quadrados, deseja-se determinar a reta $y = ax + b$ que 
melhor se ajusta aos pontos $(x_1,\,y_1)$, $(x_2,\,y_2)$, \ldots, $(x_n,\,y_n)$. Prove que:
$$a = \dfrac{\displaystyle n\sum (x_iy_i) - \sum x_i\sum y_i}{\displaystyle n\sum x_i^2 - \left(\sum x_i\right)^2},$$
$$b = \dfrac{\displaystyle \sum y_i\sum x_i^2 - \sum x_i\sum (x_iy_i)}{\displaystyle n\sum x_i^2 - \left(\sum x_i\right)^2},$$
onde em cada somatório temos $i = 1$, $2$, \ldots, $n$.
%c11a + c12b = d1
%c21a + c22b = d2
%a = (d1c22 - c12d2)/(c11c22 - c12c21)
%b = (d2c11 - c21d1)/(c11c22 - c12c21)

      \end{enumerate}
      
\begin{center}
\textbf{Gabarito}
\end{center} 
\textbf{[1]} (a) $p(x) = x^3 - x^2 + 2$ (b) $p(x) = -\dfrac{8}{3}(x-1,5)(x-2)(x-2,5) + \dfrac{25}{2}(x-1)(x-2)(x-2,5) 
- 24(x-1)(x-1,5)(x-2,5) + \dfrac{91}{6}(x-1)(x-1,5)(x-2)$ (c) $p(x) = 2 + \dfrac{9}{4}(x-1) + \dfrac{7}{2}(x-1)(x-1,5) + (x-1)(x-1,5)(x-2)$ 
\textbf{[2]} Sugestão: Comece justificando que $\dfrac{q(x)}{(x-x_i)}$, para $x\neq x_i$, é o mesmo que $\displaystyle\prod_{k=0,\,k\neq i}^n (x-x_k)$. Em seguida, 
justifique que $q'(x_i) = \displaystyle\prod_{k=0,\,k\neq i}^n (x_i-x_k)$. 
\textbf{[3]} Sugestão: Determine as expressões para $L(x)$ e $N(x)$. Em seguida, arrume essas expressões de tal modo que possamos concluir a identidade 
$L(x) = N(x)$. 
\textbf{[4]} (a) Grau $2$. (b) $\phi(x) = 1,3815x^2 - 1,9159x + 0,87686$. 
\textbf{[5]} Sugestão: Defina $\displaystyle D(a,\,b) = \sum_{i=0}^n \left[f(x_i) - \phi(x_i)\right]^2$. Em seguida, analise o sistema $\begin{cases}\dfrac{\partial D}{\partial a} = 0 \\ \\ \dfrac{\partial D}{\partial b} = 0\end{cases}$. 
\textbf{[6]} (a) $\displaystyle\int_{-1}^1 p_0(x)p_0(x)\,dx = 2$, $\displaystyle\int_{-1}^1 p_0(x)p_1(x)\,dx = 0$, \\ 
$\displaystyle\int_{-1}^1 p_0(x)p_2(x)\,dx = 0$,  $\displaystyle\int_{-1}^1 p_1(x)p_1(x)\,dx = \dfrac{2}{3}$, $\displaystyle\int_{-1}^1 p_1(x)p_2(x)\,dx = 0$, $\displaystyle\int_{-1}^1 p_2(x)p_2(x)\,dx = \dfrac{2}{5}$. 
(b) $\phi(x) = -\dfrac{13}{560} - \dfrac{17}{10}x + \dfrac{33}{14}x^2$. 
\textbf{[7]} (a) Sugestão: caso $i = j$, use a identidade 
trigonométrica $\sen^2 \alpha = \dfrac{1}{2}(1 - \cos 2\alpha)$; caso $i\neq j$, 
use a identidade $\sen \alpha\sen \beta = \dfrac{1}{2}[\cos(\alpha-\beta)-\cos(\alpha+\beta)]$; 
em ambos os casos, lembre-se que $\sen(k\pi) = 0$ quando $k\in \mathbb{Z}$. 
(b) $\phi(x) = \dfrac{2}{\pi}\sen\left(\pi x\right) - \dfrac{1}{\pi}\sen\left(2 \pi x\right) + \dfrac{2}{3\pi}\sen\left(3 \, \pi x\right) -\dfrac{1}{2\pi}\sen\left(4\pi x\right)$. 
\textbf{[8]} Sugestão: Resolva o sistema deduzido no exercício [5] considerando $g_1(x) = x$, $g_2(x) = 1$, $y_k = f(x_k)$, $k=1$, $2$, $\ldots$, $n$.
\end{document}
