\documentclass[12pt,a4paper]{article}
\usepackage[utf8]{inputenc}
\usepackage[brazil]{babel}
\usepackage{graphicx}
\usepackage{amssymb, amsfonts, amsmath}
\usepackage{color}
\usepackage{float}
\usepackage{enumerate}
\usepackage{subfigure}
\usepackage{epstopdf}

\usepackage[top=2.5cm, bottom=2.5cm, left=1.25cm, right=1.25cm]{geometry}

\DeclareMathOperator{\sen}{sen}
\DeclareMathOperator{\tg}{tg}
\DeclareMathOperator{\arcsen}{arcsen}
\DeclareMathOperator{\arctg}{arctg}
\newcommand{\limite}[3]{\displaystyle\lim_{#2 \to #3} #1}
\newcommand{\integral}[4]{\displaystyle\int_{#3}^{#4} #1\,d#2}

\begin{document}
\pagestyle{empty}

\begin{center}
\begin{tabular}{ccc}
\begin{tabular}{c}
\includegraphics[scale=0.35]{../../biblioteca/imagem/logo-ead} \\
\end{tabular} & 
\begin{tabular}{c}
Ministério da Educação \\
Universidade Federal dos Vales do Jequitinhonha e Mucuri \\
Diretoria de Educação Aberta e a Distância - DEaD \\
Disciplina: Cálculo Diferencial e Integral II\\
Prof.: Aquino\\
\end{tabular} &
\begin{tabular}{c}
\includegraphics[scale=0.25]{../../biblioteca/imagem/logo-ufvjm} \\
\end{tabular}
\end{tabular}
\end{center}

\begin{center}
 \textbf{Lista de Exercícios II}
\end{center}

 \begin{enumerate}
\item Seja $r\in\mathbb{R}^*$. Utilize uma técnica de integração conveniente para provar cada item abaixo.
\begin{enumerate}
 \item $\integral{\dfrac{1}{x^2 - r^2}}{x}{}{} = \dfrac{1}{2r}\ln\left|\dfrac{x-r}{x+r}\right| + c$.
 \item $\integral{\dfrac{1}{x^2 + r^2}}{x}{}{} = \dfrac{1}{r}\arctg\dfrac{x}{r} + c$.
\end{enumerate}

\item Supondo $n\in\mathbb{N}$ e $n\geq 2$, utilize integração por partes para provar que:
$$\integral{\cos^n x}{x}{}{}{} = \frac{1}{n}\cos^{n-1} x\sen x + \frac{n-1}{n}\integral{\cos^{n-2} x}{x}{}{}.$$
Em seguida, utilize esta fórmula para calcular $\integral{\cos^6 x}{x}{}{}$.

\item Calcule $\integral{\sec x}{x}{}{}$ seguindo o roteiro abaixo.
 \begin{itemize}
  \item Passo 1) Escreva:
  $$\integral{\sec x}{x}{}{} = \integral{\dfrac{1}{\cos x}}{x}{}{} = \integral{\dfrac{\cos x}{\cos^2 x}}{x}{}{} = \integral{\dfrac{\cos x}{1 - \sen^2 x}}{x}{}{}.$$
  \item Passo 2) Aplique a substituição $u = \sen x$ para obter:
  $$\integral{\dfrac{\cos x}{1 - \sen^2 x}}{x}{}{} = \dfrac{1}{2}\ln\left|\dfrac{1 + \sen x}{1 - \sen x}\right| + c$$
  \item Passo 3) Aplique uma simplificação algébrica conveniente para obter:
  $$\dfrac{1}{2}\ln\left|\dfrac{1 + \sen x}{1 - \sen x}\right| = \ln\left|\sec x + \tg x\right|.$$
  \item Passo 4) Conclua que:
  $$\integral{\sec x}{x}{}{} = \ln\left|\sec x + \tg x\right| + c.$$
 \end{itemize}

\item Calcule as integrais indefinidas abaixo.

\begin{tabular}{llll}
  (a) $\integral{\dfrac{2x-1}{x^{2}-4}}{x}{}{}$ & (d) $\integral{\arcsen r}{r}{}{}$ & 
  (g) $\integral{(\sen w - \cos w)^2}{w}{}{}$ & (j) $\integral{(3a^2 - 1)\left(a - \dfrac{1}{2}\right)^4}{a}{}{}$\\
  (b) $\integral{\dfrac{t^2 + 1}{\sqrt{t - 1}}}{t}{}{}$ & (e) $\displaystyle \int e^{\sqrt{s}}\, ds$ & 
  (h) $\integral{\sqrt{4 - x^2}}{x}{}{}$ & (l) $\integral{\sen y \, e^y}{y}{}{}$\\
  (c) $\integral{z^2\ln z}{z}{}{}$ & (f) $\integral{\sen^3 \alpha}{\alpha}{}{}$ & 
  (i) $\integral{\sen 7\theta \cos 3\theta}{\theta}{}{}$ & (m) $\integral{\dfrac{u - 2}{\left(u^2 - 1\right)\left(u^2 + u + 1\right)}}{u}{}{}$
\end{tabular}
\end{enumerate}

\newpage
\begin{center}
\textbf{Gabarito}
\end{center}
\textbf{[1]} (a) Sugestão: reescreva a integral original no formato $\integral{\dfrac{1}{(x + r)(x - r)}}{x}{}{}{}$ e aplique integração 
por frações parciais. 
(a) Sugestão: reescreva a integral original no formato $\dfrac{1}{r^2}\integral{\dfrac{1}{\left(\dfrac{x}{r}\right)^2 + 1}}{x}{}{}{}$ e aplique integração 
por substituição usando $u = \dfrac{x}{r}$. 
\textbf{[2]} Sugestão: reescreva a integral original no formato $\integral{\cos^{n-1} x\cos x}{x}{}{}{}$ e aplique integração por partes 
considerando $u = \cos^{n-1}x$ e $dv = \cos x \,dx$. Aplicando a fórmula, temos que: 
%\frac{1}{n}\cos^{n-1} x\sen x + \frac{n-1}{n}\integral{\cos^{n-2} x}{x}{}{}
$\integral{\cos^6 x}{x}{}{}{} = \frac{1}{6}\cos^{5} x\sen x + \frac{5}{6}\integral{\cos^{4} x}{x}{}{} = 
\frac{1}{6}\cos^{5} x\sen x + \frac{5}{6}\left(\frac{1}{4}\cos^{3} x\sen x + \frac{3}{4}\integral{\cos^2 x}{x}{}{}\right) = 
\frac{1}{6}\cos^{5} x\sen x + \frac{5}{6}\left[\frac{1}{4}\cos^{3} x\sen x + \frac{3}{4}\left(\frac{1}{2}\cos x\sen x + \frac{1}{2}\integral{1}{x}{}{}\right)\right] = 
\frac{1}{6}\cos^{5} x\sen x + \frac{5}{24}\cos^{3} x\sen x + \frac{5}{16}\cos x\sen x + \frac{5}{16}x + c$. 
\textbf{[3]} Passo 2) Sugestão: após aplicar a substituição, note que usando a técnica de frações parciais obtemos 
$\integral{\dfrac{1}{1 - u^2}}{u}{}{} = \dfrac{1}{2}\integral{\dfrac{1}{1 + u} + \dfrac{1}{1 - u}}{u}{}{}$. 
Passo 3) Sugestão: note que 
$\dfrac{1}{2}\ln\left|\dfrac{1 + \sen x}{1 - \sen x}\right| = \ln\left|\dfrac{(1 + \sen x)(1 + \sen x)}{(1 - \sen x)(1 + \sen x)}\right|^{\frac{1}{2}}$. 
\textbf{[4]} (a) $\dfrac{5}{4}\ln|x + 2| + \dfrac{3}{4}\ln|x - 2| + c$. 
(b) $\dfrac{2}{5}(t - 1)^{\frac{5}{2}} + \dfrac{4}{3}(t - 1)^{\frac{3}{2}} + 4\sqrt{t - 1} + c$. 
(c) $\dfrac{1}{3}z^3\ln z - \dfrac{1}{9}z^3 + c$. 
(d) $x\arcsen r + \sqrt{1 - r^2} + c$. 
(e) $2\left(\sqrt{s} - 1\right)e^{\sqrt{s}} + c$. 
(f) $\dfrac{1}{3}\cos^3 \alpha - \cos \alpha + c$. 
(g) $\cos^2 w + w + c$. 
(h) $\dfrac{x}{2}\sqrt{4 - x^2} + 2\,\textrm{arcsen}\,\dfrac{1}{2}x + c$. 
(i) $-\dfrac{1}{4}\left(\dfrac{1}{5}\cos 10\theta + \dfrac{1}{2}\cos 4\theta\right) + c$. 
(j) $\dfrac{3}{7}\left(a-\dfrac{1}{2}\right)^7 + \dfrac{1}{2}\left(a-\dfrac{1}{2}\right)^6 - \dfrac{1}{20}\left(a-\dfrac{1}{2}\right)^5 + c$. 
(l) $\dfrac{1}{2}(\sen y - \cos y)e^y + c$.
(m) $-\dfrac{1}{6}\ln|u - 1| + \dfrac{3}{2}\ln|u + 1| - \dfrac{2}{3}\ln\left(u^2 + u + 1\right) + \dfrac{2\sqrt{3}}{3}\arctg\left[\dfrac{\sqrt{3}}{3}(2u + 1)\right] + c$. 

\end{document}
