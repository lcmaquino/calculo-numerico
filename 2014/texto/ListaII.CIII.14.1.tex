\documentclass[12pt,a4paper]{article}
\usepackage[utf8]{inputenc}
\usepackage[brazil]{babel}
\usepackage{graphicx}
\usepackage{amssymb, amsfonts, amsmath}
\usepackage{color}
\usepackage{float}
\usepackage{enumerate}
\usepackage{subfigure}
\usepackage{epstopdf}

\usepackage[top=2.5cm, bottom=2.5cm, left=1.25cm, right=1.25cm]{geometry}

\DeclareMathOperator{\sen}{sen}
\DeclareMathOperator{\tg}{tg}
\DeclareMathOperator{\arcsen}{arcsen}
\DeclareMathOperator{\arctg}{arctg}
\newcommand{\limite}[3]{\displaystyle\lim_{#2 \to #3} #1}
\newcommand{\integral}[4]{\displaystyle\int_{#3}^{#4} #1\,d#2}
\newcommand{\parcial}[2]{\dfrac{\partial #1}{\partial #2}}

\begin{document}
\pagestyle{empty}

\begin{center}
\begin{tabular}{ccc}
\begin{tabular}{c}
\includegraphics[scale=0.35]{../../biblioteca/imagem/logo-ead} \\
\end{tabular} & 
\begin{tabular}{c}
Ministério da Educação \\
Universidade Federal dos Vales do Jequitinhonha e Mucuri \\
Diretoria de Educação Aberta e a Distância - DEaD \\
Disciplina: Cálculo Diferencial e Integral II\\
Prof.: Aquino\\
\end{tabular} &
\begin{tabular}{c}
\includegraphics[scale=0.25]{../../biblioteca/imagem/logo-ufvjm} \\
\end{tabular}
\end{tabular}
\end{center}

\begin{center}
 \textbf{Lista de Exercícios II}
\end{center}

\begin{enumerate}

\item Considerando $f(x,\,y,\,z) = \dfrac{y}{zx^2 + 1} - (z + x)\left(y^2 - z^2\right)$, calcule:

\begin{tabular}{lll}
 (a) $\dfrac{\partial f}{\partial x}$. & (b) $\dfrac{\partial f}{\partial y}$. & (c) $\dfrac{\partial f}{\partial z}$.
\end{tabular}

\item Determine o plano tangente ao gráfico das funções abaixo nos pontos indicados...

\item Calcule uma aproximação do número $\sqrt{17} + \sqrt[3]{7}$ usando um plano tangente ao gráfico da 
função definida por $f(x,\,y) = \sqrt{x} + \sqrt[3]{y}$.

\item Determine uma parametrização $\alpha(t)$ da curva de nível $c$ da função dada por $f(x,\,y) = \sqrt{x^2 + y^2}$. Em seguida, 
verifique que $\nabla f(\alpha(t)) \cdot \alpha'(t) = 0$.

\item Se $f(x,\,y,\,z) = \ln\sqrt{x^2 + y^2 + z^2}$ e $u = (x,\,y,\,z)$, verifique que $u\cdot \nabla f(u)= 1$.

\end{enumerate}

\newpage

\begin{center}
\textbf{Gabarito}
\end{center}
%\dfrac{y}{zx^2 + 1} - (z + x)\left(y^2 - z^2\right)
\textbf{[1]} (a) $\parcial{f}{x} = -\dfrac{2xyz}{\left(zx^2 + 1\right)^2} - y^2 + z^2$. 
(b) $\parcial{f}{y} = \dfrac{1}{zx^2 + 1} - 2(x + z)y$. 
(c) $\parcial{f}{z} = - \dfrac{x^2y}{\left(zx^2 + 1\right)^2} - y^2 + z^2 + 2z(x + z)$. 

\textbf{[2]} 
\textbf{[3]} 
\textbf{[4]}  
\textbf{[5]} 
\textbf{[6]} 
\textbf{[7]} 
\textbf{[8]} Sugestão: lembre-se que $(x,\,y,\,z)\cdot \left(\parcial{f}{x},\,\parcial{f}{y},\,\parcial{f}{z}\right) = x\parcial{f}{x} + y\parcial{f}{y} + z\parcial{f}{z}$. 
\end{document}
