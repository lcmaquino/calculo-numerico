\documentclass[12pt,a4paper]{article}
\usepackage[latin1]{inputenc}
\usepackage[brazil]{babel}
\usepackage{graphicx}
\usepackage{amssymb, amsfonts, amsmath}
\usepackage{color}
\usepackage{float}
\usepackage{enumerate}
%\usepackage{subfigure}
\usepackage[top=2.5cm, bottom=2.5cm, left=1.25cm, right=1.25cm]{geometry}

\newcommand{\sen}{\,\textrm{sen}\,}
\newcommand{\tg}{\,\textrm{tg}\,}

\begin{document}
\pagestyle{empty}

\begin{center}
\begin{tabular}{ccc}
\begin{tabular}{c}
\includegraphics[scale=0.25]{../../biblioteca/imagem/brasao-de-armas-brasil} \\
\end{tabular} & 
\begin{tabular}{c}
Ministério da Educação \\
Universidade Federal dos Vales do Jequitinhonha e Mucuri \\
Faculdade de Ciências Sociais, Aplicadas e Exatas - FACSAE \\
Departamento de Ciências Exatas - DCEX \\
Disciplina: Cálculo Numérico\\
Prof.: Luiz C. M. de Aquino\\
\end{tabular} &
\begin{tabular}{c}
\includegraphics[scale=0.25]{../../biblioteca/imagem/logo-ufvjm} \\
\end{tabular}
\end{tabular}
\end{center}

\begin{center}
 \textbf{Lista de Exercícios VI}
\end{center}

\begin{enumerate}
  
  \item Seja uma função $f$ da qual são conhecidos os valores descritos na tabela abaixo.
%p(x) = x^3 - x^2 + 2
\begin{center}
 \begin{tabular}{c|c|c|c|c}
  $x_i$ & 1 & 1,5 & 2 & 2,5\\ \hline
  $f(x_i)$ & 2 & 3,125 & 6 & 11,375
 \end{tabular}
\end{center}

  Determine o polinômio $p$ que interpola $f$ utilizando três maneiras:
  \begin{enumerate}
   \item resolvendo o sistema formado pelas equações $p(x_i) = f(x_i)$;
   \item escrevendo $p$ na Forma de Lagrange;
   \item escrevendo $p$ na Forma de Newton.
  \end{enumerate}

  \item Seja $p$ o polinômio na Forma de Lagrange que interpola os pontos $(x_0,\,y_0)$, $(x_1,\,y_1)$, \ldots, $(x_n,\,y_n)$. Vamos definir o 
polinômio $$q(x) = \prod_{i=0}^{n} (x-x_i).$$ Prove que $p$ pode ser escrito no seguinte formato: 
$$p(x) = \sum_{i=0}^n\frac{q(x)}{(x-x_i)q'(x_i)}y_i.$$

  \item Seja uma função $f$ da qual são conhecidos os pontos $(x_0,\,f(x_0))$ e $(x_1,\,f(x_1))$. Considere que 
$L(x)$ seja o polinômio na Forma de Lagrange que interpola $f$. Além disso, considere que $N(x)$ seja o polinômio na Forma de Newton que 
interpola $f$. Prove que $L(x)$ e $N(x)$ representam um mesmo polinômio.

\end{enumerate}

\begin{center}
\textbf{Gabarito}
\end{center} 
\textbf{[1]} (a) $p(x) = x^3 - x^2 + 2$ (b) $p(x) = -\dfrac{8}{3}(x-1,5)(x-2)(x-2,5) + \dfrac{25}{2}(x-1)(x-2)(x-2,5) 
- 24(x-1)(x-1,5)(x-2,5) + \dfrac{91}{6}(x-1)(x-1,5)(x-2)$ (c) $p(x) = 2 + \dfrac{9}{4}(x-1) + \dfrac{7}{2}(x-1)(x-1,5) + (x-1)(x-1,5)(x-2)$ 
\textbf{[2]} Sugestão: Comece justificando que $\dfrac{q(x)}{(x-x_i)}$, para $x\neq x_i$, é o mesmo que $\displaystyle\prod_{k=0,\,k\neq i}^n (x-x_k)$. Em seguida, 
justifique que $q'(x_i) = \displaystyle\prod_{k=0,\,k\neq i}^n (x_i-x_k)$. 
\textbf{[3]} Sugestão: Determine as expressões para $L(x)$ e $N(x)$. Em seguida, arrume essas expressões de tal modo que possamos concluir a identidade 
$L(x) = N(x)$.
\end{document}
