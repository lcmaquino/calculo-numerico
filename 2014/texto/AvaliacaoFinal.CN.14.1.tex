\documentclass[10pt,a4paper]{article}
\usepackage[utf8]{inputenc}
\usepackage[brazil]{babel}
\usepackage{graphicx}
\usepackage{amssymb, amsfonts, amsmath}
\usepackage{color}
\usepackage{float}
\usepackage{enumerate}
%\usepackage{subfigure}
\usepackage[top=2.5cm, bottom=2.5cm, left=1.25cm, right=1.25cm]{geometry}

\newcommand{\sen}{\,\textrm{sen}\,}
\newcommand{\tg}{\,\textrm{tg}\,}

\begin{document}
\pagestyle{empty}

\begin{center}
\begin{tabular}{ccc}
\begin{tabular}{c}
\includegraphics[scale=0.25]{../../biblioteca/imagem/brasao-de-armas-brasil} \\
\end{tabular} & 
\begin{tabular}{c}
Ministério da Educação \\
Universidade Federal dos Vales do Jequitinhonha e Mucuri \\
Faculdade de Ciências Sociais, Aplicadas e Exatas - FACSAE \\
Departamento de Ciências Exatas - DCEX \\
Disciplina: Cálculo Numérico\\
Prof.: Luiz C. M. de Aquino\\
\end{tabular} &
\begin{tabular}{c}
\includegraphics[scale=0.25]{../../biblioteca/imagem/logo-ufvjm} \\
\end{tabular}
\end{tabular}

Aluno(a): \rule{0.5\textwidth}{0.01cm} \quad Data: \rule{0.6cm}{0.01cm} / \rule{0.6cm}{0.01cm} / \rule{1.25cm}{0.01cm}
\end{center}

\begin{center}
 \textbf{Exame Final}
\end{center}

\textbf{Instruções}
\begin{itemize}
 \item Todas as justificativas necessárias na solução de cada questão devem estar presentes nesta avaliação;
 \item As respostas finais de cada questão devem estar escritas de caneta;
 \item Esta avaliação tem um total de 100,0 pontos.
\end{itemize}

\begin{enumerate}

 \item \textbf{[20 pontos]} Dado $a\in\mathbb{R}_+^*$ explique uma maneira de usar o Método da Bisseção para calcular um valor aproximado de $\sqrt{a}$ com tolerância de $10^{-5}$.

 \item \textbf{[20 pontos]} Seja $x$ um número natural não nulo qualquer. Considere que $n$ seja um quadrado perfeito mais próximo de $x$. Prove que $\sqrt{x}\approx \dfrac{x+n}{2\sqrt{n}}$. 
 (Observação: dizemos que $n$ é um quadrado perfeito se existe um natural $m$ tal que $n = m^2$.)

 \item Considere o seguinte sistema de equações:
  $$%x = -1, y = 1, z = 2.
   \begin{cases}
    2x - 4y + 8z  - w = -6 \\
    -2x  - 2y + z - 7w = -5 \\
    5x  - y + z - 2w = -2 \\
    x - 4y - z + w = 8
   \end{cases}
  $$

  \begin{enumerate}
   \item \textbf{[5 pontos]} Da forma como ele está arrumado, é recomendável usar diretamente o método de Gauss-Jaboci? Justifique sua resposta.
   \item \textbf{[15 pontos]} Arrume esse sistema de modo a utilizar o método de Gauss-Jacobi. Justifique sua arrumação. Em seguida, exiba as equações 
         utilizadas pelo método para obter uma solução aproximada desse sistema.
  \end{enumerate}
 
 \item \textbf{[15 pontos]} Seja uma função $f$ da qual são conhecidos os valores descritos na tabela abaixo.
  \begin{center}
   \begin{tabular}{c|c|c|c}
    $x_i$ & 1 & 1,5 & 2\\ \hline
    $f(x_i)$ & 2 & 3,125 & 6
   \end{tabular}
  \end{center}

  Determine o polinômio $p$ que interpola $f$ utilizando duas maneiras:
  \begin{enumerate}
   \item escrevendo $p$ na Forma de Lagrange;
   \item escrevendo $p$ na Forma de Newton.
  \end{enumerate}

 \item \textbf{[25 pontos]} Utilizando o Método dos Mínimos Quadrados deseja-se determinar $\displaystyle \phi(x) = \sum_{k=1}^n a_kg_k(x)$ que melhor se ajusta 
a uma função $f$ no intervalo $[a;\, b]$. Suponha que as funções $g_1$, $g_2$, \ldots, $g_n$ sejam escolhidas de tal forma que 
$\displaystyle\int_a^b g_i(x)g_j(x)\,dx = \begin{cases}i,\,i = j \\ 0,\,i\neq j\end{cases}$. Prove que nesse caso teremos 
$a_k = \dfrac{1}{k}\displaystyle\int_a^b f(x)g_k(x)\,dx$.


\end{enumerate}
\end{document}
